\section*{Some solutions}

\fancyhf{}
\fancyfoot[R]{\thepage}
\fancyhead[L]{\course}
\fancyhead[R]{Some solutions}
\setlength{\headheight}{14pt}

\begin{sol}{ex:taurtau}
    It is enough to apply Proposition~\ref{prop:characterisation}.
\end{sol}


\begin{sol}{ex:derived}
    First, let us prove note that $s(x,y) =(y,\sigma_y(x))$ satisfies the Yang--Baxter equation if and only if 
    \begin{align*}
        \sigma_z\sigma_y=\sigma_{\sigma_z(y)}\sigma_z.
    \end{align*}

    Note that 2) in Proposition~\ref{prop:characterisation}, implies that 
    \begin{align}\label{eq:lambdasigma}
        \lambda_x\sigma_y = \sigma_{\lambda_x(y)}\lambda_x.
    \end{align}
    Indeed, for any $x,y,z\in X$ it holds
    \begin{align*}
        \lambda_{\rho_{\lambda_y(z)}(x)}\rho_z(y)= \lambda_{\rho_{\lambda_y(z)}(x)}\lambda^{-1}_{\lambda_y(z)}\sigma_{_{\lambda_y(z)}}(y)
        \overset{1)}{=}\lambda^{-1}_{\lambda_x\lambda_y(z)}\lambda_x\sigma_{\lambda_y(z)}(y)
    \end{align*}
    and
    \begin{align*}
        \rho_{\lambda_{\rho_y(x)}(z)}\lambda_x(y) = \lambda^{-1}_{\lambda_{\lambda_x(y)}\lambda_{\rho_y(x)}(z)}\sigma_{\lambda_{\lambda_x(y)}\lambda_{\rho_y(x)}(z)}\lambda_x(y)\overset{1)}{=}\lambda^{-1}_{\lambda_x\lambda_y(z)}\sigma_{\lambda_x\lambda_y(z)}\lambda_x(y)
    \end{align*}
    Moreover, 3) in  Proposition~\ref{prop:characterisation}, implies that 
    \begin{align*}
        \sigma_z\sigma_y=\sigma_{\sigma_z(y)}\sigma_z.
    \end{align*}
    Indeed
    \begin{align*}
        \rho_z\rho_y(x) &=\lambda^{-1}_{\lambda_{\rho_y(x)}(z)}\sigma_{\lambda_{\rho_y(x)}(z)}\lambda^{-1}_{\lambda_x(y)}\sigma_{\lambda_x(y)}(x)
        \overset{\eqref{eq:lambdasigma}}{=}\lambda^{-1}_{\lambda_{\rho_y(x)}(z)}\lambda^{-1}_{\lambda_x(y)}\sigma_{\lambda_{\lambda_x(y)}\lambda_{\rho_y(x)}(z)}\sigma_{\lambda_x(y)}(x)\\
        &\overset{1)}{=}\lambda^{-1}_{\lambda_{\rho_y(x)}(z)}\lambda^{-1}_{\lambda_x(y)}\sigma_{\lambda_x\lambda_y(z)}\sigma_{\lambda_x(y)}(x).
    \end{align*}
    and
    \begin{align*}
        \rho_{\rho_z(y)}\rho_{\lambda_y(z)}(x) = 
        \lambda^{-1}_{\lambda_{\rho_{\lambda_y(z)}(x)}\rho_z(y)}\lambda^{-1}_{\lambda_x\lambda_y(z)}\sigma_{\lambda_x\lambda_{\lambda_y(z)}\rho_z(y)}\sigma_{\lambda_x\lambda_y(z)}(x)\\
        \overset{1)\&2)}{=}\lambda^{-1}_{\rho_{\lambda_{\rho_y(x)}(z)}\lambda_x(y)}\lambda^{-1}_{\lambda_{\lambda_x(y)}\lambda_{\rho_y(x)}(z)}\sigma_{\lambda_x\sigma_{\lambda_y(z)}(y)}\sigma_{\lambda_x\lambda_y(z)}(x)\\
        \overset{1)}{=}\lambda^{-1}_{\lambda_{\rho_y(x)}(z)}\lambda^{-1}_{\lambda_x(y)}\sigma_{\sigma_{\lambda_x\lambda_y(z)}\lambda_x(y)}\sigma_{\lambda_x\lambda_y(z)}(x).
    \end{align*}
    Hence, for all $x,y,z\in X$
    \begin{align*}
        \sigma_{\lambda_x\lambda_y(z)}\sigma_{\lambda_x(y)}(x)=\sigma_{\sigma_{\lambda_x\lambda_y(y)}\lambda_x(z)}\sigma_{\lambda_x\lambda_y(z)}(x)
    \end{align*}
    and the wanted equality follows.

    This prove that $s$ satisfies the Yang--Baxter equation.

    Now to prove that $r$ is bijective if and only if $s$ is non-degenerate (i.e. all $\sigma_y$ bijective). Let us first notice that $\varphi r\varphi^{-1} = s$ where $\varphi(x,y)=(x,\lambda_x(y))$. Indeed
    \begin{align*}
        \varphi r \varphi^{-1}(x,y) = \varphi r (x, \lambda^{-1}x(y))
        =\varphi(\lambda_x\lambda^{-1}x(y), \rho_{\lambda^{-1}x(y)}(x))
        =(y, \lambda_y\rho_{\lambda^{-1}x(y)}(x)) = (y,\sigma_y(x)).
    \end{align*}
    It follows that $r$ is bijective if and only if $s$ is bijective. Finally clearly $s$ is bijective if and only if $\sigma_y$ is bijective for every $y\in X$. 
\end{sol}

\begin{sol}{ex2}
    For every $x,y \in X$ let us write $\lambda_x = \id_X$ and $\rho_y(x)=x \triangleleft y$.
    We want to apply Proposition~\ref{prop:characterisation}.
    First note that clearly $\lambda_x\lambda_y =\id_X = \lambda_{\lambda_x(y)}\lambda_{\rho_y(x)}$, i.e. 1) is satisfied. 
    Moreover,$\lambda_{\rho_{\lambda_y(z)}(x)}\rho_z(y)=\rho_{\lambda_{\rho_y(x)}(z)}\lambda_x(y)$ reduce to the trivial identity $\rho_z(y)=\rho_{z}(y)$.
    Finally,  $\rho_z\rho_y(x)=\rho_{\rho_z(y)}\rho_{\lambda_y(z)}(x)$ is equivalent to $(x\triangleleft y)\triangleleft z=(x\triangleleft z)\triangleleft(y\triangleleft z)$.

    Now assume that $r$ is bijective. If $x_1,x_2\in X$ such that $\rho_y{x_1}=\rho_y(x_2)$, then $r(x_1,y)=r(x_2,y)$ and so $x_1=x_2$, i.e. $\rho_y$ is injective. Now, let $z \in X$ and let $x\in X$ such that $r(x,y) =(y,z)$. It follows that $\rho_y(x)=z$ and $\rho_y$ is bijective. 
    Similarly one obtains the converse. 
\end{sol}



\begin{sol}{ex:ef}
    Consider the map $\varphi:A\to B\times C, (x) \mapsto (x_B,-x_C)$. Clearly, $\varphi$ is bijective. Moreover, for $x,y\in A$ we have
    \begin{align*}
        \varphi(x\circ y) = \varphi(x_B+y+x_C) = \varphi(x_B+y_B+y_C+x_C) 
        =(x_B+y_B,-(y_C+x_C)) = (x_B+y_B,-x_C-y_C),
    \end{align*}
    and 
    \begin{align*}
        \varphi(x)+\varphi(y) = (x_B,-x_C) + (y_B,-y_C) = (x_B+y_B,-x_C-y_C).
    \end{align*}
    Hence $\varphi$ is an isomorphism from $(A,\circ$ to the direct product $B\times C$.
    
    Now, let $x,y,z \in A$. Then 
    \begin{align*}
        x\circ y - x + x\circ z &= x_B + y + x_C - (x_B+x_C) +x_B + z + x_C\\
        &=x_B + y + z + x_C = x \circ (y+z).
    \end{align*}
    Hence $(A,+,\circ)$ is a skew brace.
\end{sol}

\begin{sol}{ex:rho}
    Let $a,b,c\in A$. We have that
    \begin{align*}
        a\circ(a'+b) \overset{\eqref{compatibility}}{=} a \circ a' - a + a\circ b \overset{0=1}{=} 0 - a + a\circ b = \lambda_a(b).
    \end{align*}
    Hence, $\lambda_a(b)= a\circ(a'+b)$. Moreover,
    \begin{align*}
        \rho_b(a) = (\lambda_a(b))'\circ a \circ b =(a\circ(a'+b))' \circ a \circ b
        = (a'+b)'\circ b.
    \end{align*}
\end{sol}

\begin{sol}{ex:fix}
    Note that  
    \begin{align*}
        \lambda_{(a,x)}(b,y) &= - (a,x) + (a,x)\circ (b,y)\\ &= -(a,x)+(a+(-1)^xb, x+y) \\&=((-1)^xb,y).
    \end{align*}
    and hence $\Fix(A) = \{(0,0),(0,1)\}$ is not a normal subgroup of $(A,\circ)$. 
    In particular, $\Fix(A)$ is not an ideal of $A$.
\end{sol}

\begin{sol}{ex:hat}
    Let us compute
    \begin{align*}
        (x,y)=r^{-1}r(x,y) = (\hat{\lambda}_{\lambda_x(y)}\rho_y(x),\hat{\rho}_{\rho_y(x)}\lambda_x(y)).
    \end{align*}
    It follows that 
    \begin{align*}
        (x,\lambda^{-1}_x{y})=(\hat{\lambda}_y\rho_{\lambda^{-1}_x(y)}(x),\hat{\rho}_{\rho_{\lambda^{-1}_x(y)(x)}}(y)),
    \end{align*}
    hence $\hat{\lambda}^{-1}_y(x)=\rho_{\lambda^{-1}_x(y)}(x)$ and $\lambda^{-1}_x(y)=\hat{\rho}_{\hat{\lambda}^{-1}_y(x)}(y)$. Similarly
    \begin{align*}
        ({\rho^{-1}_y(x)},y)=(\hat{\lambda}_{\lambda_{\rho^{-1}_y(x)}(y)}(x),\hat{\rho}_{x}\lambda_{\rho^{-1}_y(x)}(y)),
    \end{align*}
    hence $\hat{\rho}^{-1}_x(y)=\lambda_{\rho^{-1}_y(y)}(y)$ and $\rho^{-1}_y(x) = \hat{\lambda}_{\hat{\rho}^{-1}_x(y)}(x)$.
\end{sol}