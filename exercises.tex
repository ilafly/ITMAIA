\documentclass[12pt]{amsproc}

\newcommand{\course}{Skew braces and solutions to the Yang--Baxter equation}
\usepackage[foot]{amsaddr} 
\usepackage{hyperref}
\usepackage{listings}
\usepackage{tikz-cd}
\usepackage{datetime}
\usepackage{mathptmx}
\usepackage{amssymb}
\usepackage{newtxtext}
\usepackage{mathtools}
\usepackage{stmaryrd}
\usepackage{mdframed}
\usepackage{textcomp}

\usepackage[margin=1in,footskip=.25in]{geometry}

\overfullrule=1mm

\lstdefinelanguage{Julia}%
  {morekeywords={abstract,break,case,catch,const,continue,do,else,elseif,%
      end,export,false,for,function,immutable,import,importall,if,in,%
      macro,module,otherwise,quote,return,switch,true,try,type,typealias,%
      using,while},%
   sensitive=true,%
   alsoother={$},%
   morecomment=[l]\#,%
   morecomment=[n]{\#=}{=\#},%
   morestring=[s]{"}{"},%
   morestring=[m]{'}{'},%
}[keywords,comments,strings]%

\definecolor{background}{HTML}{F5F5F5}
\definecolor{jlstring}{HTML}{880000}%          % julia's strings
\definecolor{jlbase}{HTML}{444444}%            % julia's base color
\definecolor{jlkeyword}{HTML}{444444}%         % julia's keywords
\definecolor{jlliteral}{HTML}{78A960}%         % julia's literals
\definecolor{jlbuiltin}{HTML}{397300}%         % julia's built-ins
\definecolor{jlmacros}{HTML}{1F7199}%          % julia's macros
\definecolor{jlfunctions}{HTML}{444444}%       % julia's functions
\definecolor{jlcomment}{HTML}{888888}%         % julia's comments
\definecolor{jlstring}{HTML}{880000}%          % julia's strings


\lstset{%
    language         = Julia,
    basicstyle       = \color{jlstring}\ttfamily\scriptsize,
    backgroundcolor  = \color{background},
    keywordstyle     = \color{jlkeyword},
    stringstyle      = \color{jlstring},
    commentstyle     = \color{jlcomment},
    showstringspaces = false,
    columns=fixed,
}

\renewcommand\sectionname{Lecture}
\renewcommand\subsectionname{\S}

% para enumerar
\renewcommand{\labelenumi}{\textbf{\arabic{enumi})}}

\usepackage[most]{tcolorbox}

\newtcolorbox{mybox}{
enhanced,
boxrule=0pt,frame hidden,
%borderline west={4pt}{0pt}{black},
colback={gray!20},
sharp corners
}

\makeindex             


\newcommand{\Hom}{\operatorname{Hom}}
\newcommand{\id}{\operatorname{id}}
\newcommand{\Aut}{\operatorname{Aut}}
\newcommand{\Inn}{\operatorname{Inn}}
\newcommand{\End}{\operatorname{End}}
\newcommand{\Ret}{\operatorname{Ret}}
\newcommand{\Soc}{\operatorname{Soc}}
\newcommand{\Fix}{\operatorname{Fix}}
\newcommand{\Sym}{\operatorname{Sym}}
\newcommand{\Ann}{\operatorname{Ann}}





\newtheorem{theorem}{Theorem}[section]
\newtheorem{lemma}[theorem]{Lemma}
\newtheorem{proposition}[theorem]{Proposition}
\newtheorem{corollary}[theorem]{Corollary}

\theoremstyle{definition}
\newtheorem{definition}[theorem]{Definition}
\newtheorem{convention}{Convention}
\newtheorem{example}[theorem]{Example}
\newtheorem{examples}[theorem]{Examples}
\newtheorem{xca}[theorem]{Exercise}
%\newmdtheoremenv{exercise}[theorem]{Exercise}
\newtheorem{remark}[theorem]{Remark}

\theoremstyle{remark}
\newtheorem*{claim}{Claim}

\newenvironment{sol}[1]
{\renewcommand{\qedsymbol}{}\begin{proof}[\ref{#1}]}
  {\end{proof}}

\newenvironment{exercise}
  {\begin{mybox}\begin{xca}}
  {\end{xca}\end{mybox}}
  
\numberwithin{equation}{section}

\makeindex

\title{\course}
\author{Ilaria Colazzo}
\address{University of Exeter -- Exeter (UK)}
\email{ilariacolazzo@gmail.com}
\thanks{}
\date{}

\makeatletter

\usepackage{fancyhdr}
\pagestyle{fancy}
\fancyhf{}
\fancyfoot[R]{\thepage}
\fancyhead[L]{\course}
\fancyhead[R]{Lecture \thesection}
\setlength{\headheight}{14pt}


\usepackage{tikz}
\usetikzlibrary{braids}


\usepackage{imakeidx}
\makeindex[program=makeindex,columns=2,intoc=true,options={-s index_style.ist}]


\begin{document}
\maketitle %This command prints the title based on information entered above
\setcounter{section}{1}
\section{22/02/2024}

\subsection{Useful definitions and results}

\begin{definition}
        A \emph{set-theoretic solution to the Yang--Baxter equation} is a pair $(X,r)$ where $X$ is a non-empty set and $r: X\times X \to X \times X$ is a map such that
        \begin{align}\label{eq:YBE}
            (r\times \id)(\id \times r)(r\times \id) = (\id \times r)(r\times \id)(\id \times r)
        \end{align}
\end{definition}

\begin{definition}\index{Set-theoretic solution!Finite}\index{Set-theoretic solution!Non-degenerate}
        Let $(X,r)$ be a set-theoretic solution to the  Yang--Baxter equation. We say that 
        \begin{itemize}
            \item $(X,r)$ is \emph{bijective} if $r$ is bijective.
            \item $(X,r)$ is \emph{finite} if $X$ is finite.
            \item $(X,r)$ is \emph{non-degenerate} if $\lambda_x,\rho_x$ are bijective for all $x\in X$.
        \end{itemize}
    \end{definition}

\begin{convention}
        From now on, a \emph{solution}  will always mean a bijective non-degenerate set-theoretic solution to the Yang--Baxter equation.
\end{convention}

\begin{definition}
         A \emph{skew (left) brace} is a triple $(A,+,\circ)$, where 
        $(A,+)$ and $(A,\circ)$ 
    	are (not necessarily abelian) 
    	groups and 
    	\begin{align}\label{compatibility}
    	    a\circ(b+c)=(a\circ b)-a+(a\circ c)
    	\end{align}
    	holds for all $a,b,c\in A$. 
    
    The groups $(A,+)$ and $(A,\circ)$ are respectively the \emph{additive} and \emph{multiplicative} group	of the skew brace $A$.
\end{definition}

\begin{fact}
    Let $A$ be a skew brace. The following statements hold.
        \begin{enumerate}
            \item $1=0$, where $0$ denotes the identity of $(A,+)$ and by $1$ the identity of $(A,\circ)$.
            \item $-(a\circ b) = - a + a\circ(- b) - a$, for all $a,b \in A$.
        \end{enumerate}
\end{fact}

    \begin{fact}\label{prop:lambda}
        Let $A$ be a skew brace. For each $a\in A$, the map 
        \begin{align*}
            \lambda_a:A\to A, b\mapsto -a + a \circ b
        \end{align*}
    is an automorphism of $(A,+)$. 
    
    Moreover, the map $\lambda: (A,\circ) \to \Aut(A,+), a \mapsto \lambda_a$, is a group homomorphism.
    \end{fact}



    \begin{fact}\label{prop:rho}
        Let $A$ be a brace. For each $a\in A$, the map
        \begin{align*}
            \rho_b\colon A\to A,\quad
            \mapsto (\lambda_a(b))'\circ a\circ b,
        \end{align*}
        is bijective. Moreover, the map 
        $\rho\colon (A,\circ)\to\Sym(A)$, $b\mapsto\rho_b$, satisfies $\rho_c\rho_b=\rho_{b\circ c}$, for all $b,c\in A$. 
    \end{fact}

    \begin{definition}
        An \emph{ideal} of a skew brace $A$ is a strong left ideal $I$ of $A$ such that    $(I,\circ)$ is a normal subgroup of $(A,\circ)$.
    \end{definition}

\subsection{Selection of problems}\mbox{ }
    
It is possible to prove the isomorphism theorems for skew braces. (See Exercises~\ref{ex:thmiso1}--\ref{ex:thmiso4}).

\begin{xca}
     A map $f:A\to B$ between two skew braces $A$ and $B$ is a \emph{homomorphism} of skew braces if $f(a + b)= f(a) +f(b)$ and $f(a\circ b)= f(a)\circ f(b)$, for all $a,b\in A$.The \emph{kernel} of $f$ is
    \begin{align*}
        \ker f = \{a\in A\colon f(a)=0\}.
    \end{align*}
    
    Let $f:A\to B$ be a homomorphism of two skew braces $A$ and $B$. 
    Prove that $\ker f$ is an ideal of $A$.
\end{xca}

\begin{xca}\label{ex:thmiso1}
    Let $f : A\to B$ be a homomorphism of skew braces. Prove that $A/\ker f \cong f(A)$.
\end{xca}

\begin{xca}\label{ex:thmiso2}
    Let $A$ be a skew brace and let $B$ be a subbrace of $A$. Prove that if $I$ is an ideal of $A$, then $B\circ I$ is a subbrace of $A$, $B\cap I$ is an ideal of $B$ and $(B\circ I)/I \cong B/(B\cap I)$.
\end{xca}

\begin{xca}\label{ex:thmiso3}
    Let $A$ be a skew brace and $I$ and $J$ be ideals of $A$. Prove that if $I\subseteq J$, then $A/J\cong (A/I)/(J/I)$.
\end{xca}

\begin{xca}\label{ex:thmiso4}
    Let $A$ be a skew brace and let $I$ be an ideal of $A$. Prove that there is a bijective correspondence between (left) ideals of $A$ containing $I$ and (left) ideals of $A/I$.
\end{xca}

\begin{xca}
    Let $A$ be a skew brace and $I$ be a characteristic subgroup of the additive. Prove that $I$ is a left ideal of $A$.
\end{xca}

Let us get some concrete examples of skew braces.

\begin{xca}
    Let $p$ be a prime number and let  $A=\mathbb{Z}/(p^2)$ the ring of integers modulo $p^2$. Prove that $A$ with respect to the usual sum and the operation given by $x \circ y = x+y+pxy$ is a skew brace.
\end{xca}

\begin{xca}[The semidirect product]
    Let $A,B$ be skew braces. 
    Let $\alpha: (B,\circ)\to\Aut(A,+,\circ)$ be a homomorphism of groups. 
    Define two operations on $A\times B$ by
    \begin{align*}
        (a,x)+(b,y) &= (a+b,x+y)\\
        (a,x)\circ(b,y) &= (a\circ\alpha_{x}(b), x\circ y),
    \end{align*}
    for all $a,b\in A$ and $x,y\in B$. Prove that $(A\times B, +,\circ)$ is a skew brace. 
    
    This skew brace is the \emph{semidirect product} of the skew brace $A$ by $B$ via $\alpha$, and it is denoted by $A\rtimes_{\alpha}B$.
\end{xca}

\begin{xca}\label{ex:ef}
        Let $(A,+)$ be a group with an \emph{exact factorisation} through the subgroups $B$ and $C$ (i.e. $B$ and $C$ are subgroups of $A$ such that $B\cap C=\{ 0\}$ and $A=B+C$). 
        This means that each $x\in A$ can be written in a unique way as $x=x_B+x_C$, for some $x_B\in B$ and $x_C\in C$.
        Set
        \begin{align*}
		    x\circ y&=x_B+y+x_C.
	    \end{align*}
        Prove that
        \begin{enumerate}
            \item $(A,\circ)$ is a group isomorphic to $B\times C$, the direct product of $B$ and $C$.
            \item $(A,+,\circ)$ is a  skew brace.
        \end{enumerate}
    \end{xca}

\subsection{More exercises}\mbox{}

\begin{xca}\label{ex:taurtau}
    Let $(X,r)$ be a set-theoretic solution to the Yang--Baxter equation. Define for all $x,y \in X$
    \begin{align*}
        \bar{r}(x,y) = \tau r \tau (x,y) = (\rho_x(y),\lambda_y(x)).
    \end{align*}
    Then $(X,\bar{r})$ is a set-theoretic solution to the Yang--Baxter equation.
\end{xca}

\index{Shelf}\index{Rack}
A \emph{(right) shelf} is a pair $(X,\triangleleft)$ where $X$ is a non-empty set and $\triangleleft$ is a binary operation such that 
    \begin{align*}
        (x\triangleleft y)\triangleleft z=(x\triangleleft z)\triangleleft(y\triangleleft z).
    \end{align*}
    If, in addition, the maps $\rho_y:X \to X, x \mapsto x\triangleleft y$ are bijective for all $y\in X$, then $(X,\triangleleft)$ is called a \emph{(right) rack}.
    
\begin{xca}\label{ex2}
     Let $X$ be a non-empty set.
     Let $\triangleleft: X\times X \to X$ be a binary operation and define $r: X\times X \to X\times X$ such that $r(x,y)= (y,x \triangleleft y)$. Then $r$ satisfies equation~\ref{eq:YBE} if and only if $(x\triangleleft y)\triangleleft z=(x\triangleleft z)\triangleleft(y\triangleleft z)$ holds for all $x,y,z \in X$. 
     Moreover, $r$ is bijective if and only if the maps $\rho_y:X\to X, x \mapsto x\triangleleft y$ are bijective.
\end{xca}

\begin{xca}
    Let $G$ be a group. 
    Prove that $G$ with respect to the binary operation $\triangleleft$ defined by $x\triangleleft y =y^{-1}xy$ is a rack.
\end{xca}

\begin{xca}
    Let $(X,r)$ be a solution. Define 
    \begin{align*}
        x\triangleleft y = \lambda_y\rho_{\lambda_x^{-1}(xy}(x).
    \end{align*}
    Prove that $(X,\triangleleft)$ is a shelf.
\end{xca}


\begin{xca}
    Let $A$ be a skew brace. Prove that
    \begin{align*}
        \rho_b(a) = \lambda^{-1}_{\lambda_a(b)}(-(a\circ b) +a+a\circ b)
    \end{align*}
\end{xca}

\begin{xca}
    Let $(A,+)$ be a (not necessarily abelian) group. 
    \begin{enumerate}
        \item Prove that a structure of skew brace over $A$ is equivalent to an operation $A\times A \to A$ $(a,b)\mapsto a\ast b$, such that
        \begin{align*}
            a \ast (b+c) = a\ast b + b + a\ast c - b
        \end{align*}
        holds for all $a,b,c \in A$ and the operation $a\circ b = a+ a\ast b + c$ turns $A$ into a group.
        \item Deduce that radical rings are examples of skew braces. 
        \end{enumerate}
\end{xca}

\begin{xca}
    Let $A$ be a skew brace and $a\ast b = \lambda_a(b)-b = -a+a\circ b - b$. Prove the following identities:
    \begin{enumerate}
        \item $a\ast (b+c) = a\ast b + b +a\ast c -b$.
        \item $(a\circ b)\ast c = (a\ast(b\ast c)) + b\ast c + a\ast c$.
    \end{enumerate}
\end{xca}

\begin{xca}
    Let $(A,+,\circ)$ be a triple, where $(A,+)$ and $(A,\circ)$ are groups, and $\lambda: A \to \Sym(A)$, $a\mapsto \lambda_a$ with $\lambda_a(b)=-a+a\circ b$. Prove that the following statements are equivalent:
    \begin{enumerate}
        \item $(A,+,\circ)$ is a skew brace.
        \item $\lambda_a\lambda_b(c)=\lambda_{a\circ b}(c)$, for all $a,b,c\in A$.
        \item $\lambda_a(b+c) = \lambda_a(b)+\lambda_a(c)$, for all $a,b,c\in A$.
    \end{enumerate}
\end{xca}

\begin{xca}\label{ex:sd}
    	Let $(A,+)$ and $(M,+)$ be groups and let $\alpha\colon A\to\Aut(M)$ be a
    	group homomorphism. Prove that $M\times A$ with 
    	\begin{align*}
        	(x,a)+(y,b)&=(x+y,a+b),
        	\\
        	(x,a)\circ (y,b)&=(x+\alpha_a(y),a+b)
    	\end{align*}
    	is a skew brace. Similarly, prove that $M\times A$ with
    	\begin{align*}
        	(x,a)+(y,b)&=(x+\alpha_a(y),a+b),\\
        	(x,a)\circ (y,b)&=(x+y,b+a)
    	\end{align*}
    	is a skew brace. 
    \end{xca}

    


\begin{xca}\label{ex:fix}
    Consider the semidirect product $A= \mathbb{Z}/(3) \rtimes \mathbb{Z}/(2)$ of the trivial
    skew braces $\mathbb{Z}/(3)$ and $\mathbb{Z}/(2)$ via the non-trivial action of $\mathbb{Z}/(2)$ over $\mathbb{Z}/(3)$.
    
    Prove that $\Fix(A)=\{b \in B\colon \lambda_x(b)=b,\ \forall b\in A\}$ is not an ideal of $A$.
\end{xca}






\end{document}