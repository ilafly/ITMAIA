\section*{Appendix}

\fancyhf{}
\fancyfoot[R]{\thepage}
\fancyhead[L]{\course}
\fancyhead[R]{Appendix}
\setlength{\headheight}{14pt}



\subsection{Radical rings}
    \begin{definition}
         A non-empty set $R$ with two binary operations the addition~$+$ (addition) and the multiplication~$\cdot$ is a \emph{ring}\index{Ring} if
    \begin{itemize}
        \item $(R,+)$ is an abelian group,
        \item $(R,\cdot)$ is a semigroup (i.e. $\cdot$ is associative),
        \item The multiplication is distributive with respect to the addition, i.e.
        \begin{align*}
            a\cdot (b+c)&=(a\cdot b)+(a\cdot c)&\text{(left distributivity)}\\
            (b+c)\cdot a&=(b\cdot a)+(c\cdot a)&\text{(right distributivity)}
        \end{align*}
        for all $a,b,c\in R$.
    \end{itemize}
    A ring $(R,+,\cdot)$ is \emph{unitary} if there is an element $1$ in $R$ such that $a\cdot 1= 1\cdot a=a$ for all $a \in R$ (i.e., $1$ is the \emph{multiplicative identity}).
    \end{definition}

    Let $R$ be a non-unitary ring. Consider $R_1 = \mathbb{Z}\times R$ with the addition defined component-wise and multiplication
    \begin{align*}
        (k,a)(l,b)=(kl,kb+la+ab)
    \end{align*}
    for all $k,l\in \mathbb{Z}$ and $a,b\in R$.
    
    Then $R_1$ is a ring and $(1, 0)$ is its multiplicative identity.
    
    Note that $\{0\}\times R$ is isomorphic to $R$ as non-unitary rings.

    \begin{exercise}\label{ex:invertible elements}
         Let $R$ be a non-unitary ring. Consider $R_1 = \mathbb{Z}\times R$ as before. If $(k,x)\in R_1$ is invertible, then $k \in\{1,-1\}$.
    \end{exercise}

   \begin{definition}
        Let $R$ be a unitary ring.  The \emph{(Jacobson) radical} $J(R)$ of $R$ is defined as the intersection of all maximal left ideals\footnote{A \emph{left ideal} of $R$ is an additive subgroup $I$ of $R$ such that $ax\in I$ for all $a\in R$ and $x\in I$.} of $R$.
   \end{definition}

   \begin{exercise}\label{ex:Jideal}
       Let $R$ be a unitary ring. 
       \begin{enumerate}
           \item Prove that $J(R)$ in an ideal of $R$.
           \item Prove that  $x\in J(R)$ if and only if $1 + rx$ is invertible for all $r\in R$.
       \end{enumerate}
   \end{exercise}

   \begin{definition}
       A non-unitary ring $R$ is a  \emph{(Jacobson) radical ring} if it is isomorphic
        to the Jacobson radical of a unitary ring. 
   \end{definition}

    \begin{proposition}
        Let $R$ be a non-unitary ring. The following statements are equivalent.
    \begin{enumerate}
        \item $R$ is a radical ring.
        \item For all $a\in R$ there exists a unique $b\in R$ such that $a+b+ab=a+b+ba= 0$.
        \item $R$ is isomorphic to $J(R_1)$.
    \end{enumerate}
    \end{proposition}

    \begin{proof}
        Let us first prove that 1) implies 2). 
        Let $M$ be a unitary ring such that $R$ is isomorphic to its Jacobson radical $J(M)$ and let $\psi: R \to M$ be a homomorphism such that $\psi(R)$ is isomorphic to $J(M)$. 
        Now, if $a\in R$, then $\psi(a)\in J(M)$. By Exercise~\ref{ex:Jideal}, $1+\psi(a)$ is invertible, i.e. there exists $c\in M$ such that 
        \begin{align*}
            (1+\psi(a))(1+c) = 1 = (1+c) (1+\psi(a)).
        \end{align*}
        It follows that $c\in J(M)$, i.e. $c=\psi(b)$ for some $b\in R$. Moreover, since $\psi$ is a homomorphism
        \begin{align*}
            1 &= (1+\psi(a))(1+c) = (1+\psi(a))(1+\psi(b)) \\&= 1 + \psi(a)+\psi(b)+\psi(a)\psi(b)
            = 1 + \psi(a+b +ab)
        \end{align*}
        and 
        \begin{align*}
            1 &= (1+c)(1+\psi(a)) = (1+\psi(b))(1+\psi(a)) \\&= 1 + \psi(b)+\psi(a)+\psi(b)\psi(a)
            = 1 + \psi(a+b+ba).
        \end{align*}
        Hence, 2) holds. 

        Now let us prove 2) implies 3). Let $a\in R$, we aim to prove that $(1,a)\in R_1$ is invertible. By 2) there exists $b\in R$ such that 
        \begin{align*}
            (1,a)(1,b) = (1, a+b+ab) = (1,0)\\
            (1,b)(1,a) = (1, b+a+ba) = (1,0).   
        \end{align*}
        Now, consider $(k,a)\in J(R_1)$. We want to prove that $k=0$, i.e. $J(R_1)\subseteq \{0\}\times R$. Since $(k,a)\in J(R_1)$ follows that $(1,0)+(3,0)(k,a)=)(1+3k,3a)$ is invertible by Exercise~\ref{ex:Jideal}, and so $k=0$. Therefore $J(R_1)\subseteq \{0\}\times R$. Moreover, let $(0,R)\in\{0\}\times R$. then 
        \begin{align*}
            (1,0) + (k,a) (0,x) = (1,0)+(0,kx+ka) = (1, kx+ka)
        \end{align*}
        which is invertible. So $(0,x)\in J(R_1)$.
        Finally the implication 3) implies 1) is trivially true.
    \end{proof}

    \begin{definition}
         Let $R$ be any ring. Define on $R$ the binary operation $\circ$ called the \emph{adjoint multiplication} of $R$
        \begin{align*}
            a\circ b = a+b+ab,
        \end{align*}
        for all $a,b\in R$. 
    \end{definition}

    \begin{lemma}\label{lem:adjoint}
        Then $(R,\circ)$ is a monoid with neutral element $0$. 
    \end{lemma}

    \begin{exercise}
        Prove Lemma~\ref{lem:adjoint}.
    \end{exercise}

    \begin{convention}
        If $a\in R$ is invertible in the monoid $(R, \circ)$, 
        we will denote by $a'$ its inverse.
    \end{convention}

    \begin{examples}\mbox{}
        \begin{enumerate}
            \item Let $p$ be a prime and let $A = \mathbb{Z}/(p^2) = \mathbb{Z}/p^2\mathbb{Z}$ be the ring of integers modulo $p^2$. Then $(A,+)$ with a new multiplication $\ast$ defined by $a\ast b = pab$ is a radical ring. In this case, $a\circ b = a+b+pab$, and $a' =-a+pa^2$. 
            \item Let $n$ be an integer such that $n>1$. Let 
                \begin{align*}
                    A=\left\{\dfrac{nx}{ny+1}\colon x,y \in \mathbb{Z}\right\}\subseteq \mathbb{Q}.
                \end{align*}
                $A$ is a (non-unitary) subring of $\mathbb{Q}$. In fact, $A$ is a radical ring. A straightforward computation shows
                \begin{align*}
                    \left(\dfrac{nx}{ny+1}\right)' = \dfrac{-nx}{n(x+y)+1}.
                \end{align*}
        \end{enumerate}
    \end{examples}


    \subsection{An intriguing connection between group actions and solutions}
    The following theorem is the core result of the paper \cite{LYZ00} by Lu, Yan Zhu.

    \begin{theorem}\label{thm:LYZ}
         Let $G$ be a group, let $\lambda: G\times G \to G, (x,y)\mapsto \lambda_x(y)$ a left group action of $G$ on itself as a set and $\rho: G\times G \to G, (x,y)\mapsto \rho_y(x)$ a right group action of $G$ on itself as a set. If the ``compatibility'' condition 
        \begin{align}\label{eq:LYZ}
            uv = \lambda_u(v)\rho_v(u)
        \end{align}
        holds, then $(G,r)$, where
        \begin{align*}
            r:G\times G \to G\times G, \qquad (x,y)\mapsto (\lambda_x(y),\rho_y(x))
        \end{align*}
        is a solution.
    \end{theorem}

    \begin{proof}
    Let us write $r_1=r\times \id$ and $r_2=\id \times r$,  
    \begin{align*}
         r_1r_2r_1(x,y,z)&= (\lambda_{\lambda_x(y)}\lambda_{\rho_y(x)}(z),\rho_{\lambda_{\rho_y(x)}(z)}\lambda_x(y),\rho_z\rho_y(x)) \\&=(u_1,v_1,w_1),
    \end{align*}
    and
    \begin{align*}
        r_2r_1r_2(x,y,z)&=(\lambda_x\lambda_y(z),\lambda_{\rho_{\lambda_y(z)}(x)}\rho_z(y),\rho_{\rho_z(y)}\rho_{\lambda_y(z)}(x))\\&=(u_2,v_2,w_2).
    \end{align*}
    Then we obtain
    \begin{align*}
        u_1v_1w_1 &= \lambda_{\lambda_x(y)}\lambda_{\rho_y(x)}(z)\rho_{\lambda_{\rho_y(x)}(z)}\lambda_x(y)\rho_z\rho_y(x)\\
        &\overset{\eqref{eq:LYZ}}{=}\lambda_x(y)\lambda_{\rho_y(x)}(z)\rho_z\rho_y(x)\\
        &\overset{\eqref{eq:LYZ}}{=}\lambda_x(y)\rho_y(x)z\\
        &\overset{\eqref{eq:LYZ}}{=}xyz
    \end{align*}
    and, similarly 
    \begin{align*}
        u_2v_2w_2 &=\lambda_x\lambda_y(z)\lambda_{\rho_{\lambda_y(z)}(x)}\rho_z(y)\rho_{\rho_z(y)}\rho_{\lambda_y(z)}(x)\\
        &\overset{\eqref{eq:LYZ}}{=}\lambda_x\lambda_y(z)\rho_{\lambda_y(z)}(x)\rho_z(y)
        \\
        &\overset{\eqref{eq:LYZ}}{=}x\lambda_y(z)\rho_z(y)\\
        &\overset{\eqref{eq:LYZ}}{=} xyz.
    \end{align*}
    Hence
    \begin{align}\label{eq:LYXproduct}
        u_1v_1w_1 =  xyz = u_2v_2w_2.
    \end{align}
    Moreover, since $\lambda$ is a left action of $G$ on itself, we  get
    \begin{align*}
        u_1=\lambda_{\lambda_x(y)}\lambda_{\rho_y(x)} (z) 
        = \lambda_{\lambda_{x}(y)\rho_y(x)}(z) 
        \overset{\eqref{eq:LYZ}}{=}\lambda_{xy}(z) = \lambda_x\lambda_y(z)=u_2.
    \end{align*}
    Similarly, since $\rho$ is a right action
    \begin{align*}
        w_2 = \rho_{\rho_z(y)}\rho_{\lambda_y(z)}(x) = \rho_{\lambda_y(z)\rho_z(y)} (x)\overset{\eqref{eq:LYZ}}{=} \rho_{yz}(x) = \rho_z\rho_y(x) = w_1.
    \end{align*}
    From \eqref{eq:LYXproduct} and $G$ being a group it follows that also $v_1=v_2$.
    Moreover, $\lambda_x$ and $\rho_x$ are bijective maps by assumption. 
    It is left to prove that $r$ is bijective.
    First let us write $r(u,v)=(x,y)$, hence $\lambda_u(v)=x$, $\rho_v(u)=y$, and $uv=xy$. Now, since $\lambda$ is an action and in particular $\lambda^{-1}_v=\lambda_{v^{-1}}$, we get 
    \begin{align*}
        \lambda_y(v^{-1})u = \lambda_y(v^{-1})\rho^{-1}_v(y) = \lambda_y(v^{-1})\rho_{v^{-1}}(y) \overset{\eqref{eq:LYZ}}{=} yv^{-1} = x^{-1}u = (\lambda_u(v))^{-1}u,
    \end{align*}
    and so 
    \begin{align}\label{eq:lambdainv}
        (\lambda_u(v))^{-1}=\lambda_{\rho_v(u)}(v^{-1}).
    \end{align}
    Similarly, expanding $v\rho_x(u^{-1})$ one proves
    \begin{align}\label{eq:rhoinv}
        (\rho_v(u))^{-1}=\rho_{\lambda_u(v)}(u^{-1}).
    \end{align}
    Define
    \begin{align*}
        r'(x,y)=((\rho_{x^{-1}}(y^{-1}))^{-1},(\lambda_{y^{-1}}(x^{-1}))^{-1}).
    \end{align*}
    Then
    \begin{align*}
        rr'(x,y) &= (\lambda_{(\rho_{x^{-1}}(y^{-1}))^{-1}}((\lambda_{y^{-1}}(x^{-1}))^{-1}), \rho_{(\lambda_{y^{-1}}(x^{-1}))^{-1}}((\rho_{x^{-1}}(y^{-1}))^{-1}))\\
        &\overset{\eqref{eq:lambdainv}\&\eqref{eq:rhoinv}}{=}(\lambda^{-1}_{\rho_{x^{-1}}(y^{-1})}\lambda_{\rho_{x^{-1}}(y^{-1})}(x),\rho^{-1}_{\lambda_{y^{-1}}(x^{-1})}\rho_{\lambda_{y^{-1}}(x^{-1})}(y)) \\
        &=(x,y).
    \end{align*}
    And
    \begin{align*}
        r'r(x,y) &= ((\rho_{(\lambda_x(y))^{-1}}((\rho_y(x))^{-1}))^{-1},(\lambda_{(\rho_y(x))^{-1}}((\lambda_x(y))^{-1}))^{-1})\\
        &\overset{\eqref{eq:lambdainv}\&\eqref{eq:rhoinv}}{=} ( (\rho^{-1}_{\lambda_x(y)}\rho_{\lambda_x(y)}(x^{-1}))^{-1}, (\lambda^{-1}_{\rho_y(x)}\lambda_{\rho_y(x)}(y^{-1}))^{-1})\\
        &=((x^{-1})^{-1},(y^{-1})^{-1}) = (x,y).
    \end{align*}
\end{proof}



\subsection{The retraction of a solution.}

Let $(X,r)$ be a solution and define on $X$ the following relation
\begin{align*}
    x\sim y \quad \iff \quad \lambda_x=\lambda_y \text{ and } \rho_x=\rho_y.
\end{align*}

Let $\bar{X}=X/\sim$ denote the set of equivalence classes and let $[x]$ denote the class of $x$.

\begin{lemma}\label{lem:hat}
    Let $(X,r)$ be a solution and write $r^{-1}(x,y)= (\hat{\lambda}_x(y),\hat{\rho}_y(x))$.
    Then 
    \begin{align}
        \label{eq:hat1}\hat{\lambda}^{-1}_y(x)=\rho_{\lambda^{-1}_x(y)}(x),\\
        \label{eq:hat2}\lambda^{-1}_x(y)=\hat{\rho}_{\hat{\lambda}^{-1}_y(x)}(y),\\
        \label{eq:hat3}\hat{\rho}^{-1}_x(y)=\lambda_{\rho^{-1}_y(y)}(y),\\
        \label{eq:hat4}\rho^{-1}_y(x) = \hat{\lambda}_{\hat{\rho}^{-1}_x(y)}(x).
    \end{align}
\end{lemma}

\begin{exercise}
    \label{ex:hat}
    Prove Lemma~\ref{lem:hat}.
\end{exercise}

\begin{theorem}
    Let $(X,r)$ be a solution. Then $r$ induce a solution $\bar{r}$ on $\bar{X}$ by
    \begin{align*}
        \bar{r}([x],[y])= ([\lambda_x(y)], [\rho_y(x)]),
    \end{align*}
    for all $x,y\in X$.
\end{theorem}

\begin{proof}
    Omitted.
\end{proof}

\begin{definition}
    Let $(X,r)$ be a solution. The solution $\operatorname{Ret}(X,r) =(\bar{X},\bar{r})$ induced by the equivalence relation $\sim$ is the \emph{retraction} of $(X,r)$.

    We define inductively $\operatorname{Ret}^{0}(X,r)=(X,r)$, $\operatorname{Ret}^1(X,r)=\operatorname{Ret}(X,r)$ and 
    \begin{align*}
        \operatorname{Ret}^{n+1}(X,r) = \operatorname{Ret}(\operatorname{Ret}^n(X,r)), \quad n\geq 1.
    \end{align*}
\end{definition}


\begin{definition}
     A solution $(X,r)$ is said to be a \emph{multipermutation solution of level} $n$, if $n$ is the smallest non-negative integer such that $|\operatorname{Ret}^{n}(X,r)| = 1$. 
\end{definition}
  
    
\begin{definition}
    The solution $(X,r)$ is said to be \emph{irretractable} if $\operatorname{Ret}(X,r)=(X,r)$. 
\end{definition}

\begin{example}
    The trivial solution $(X,\tau)$ over a set with one element is a multipermutation solution of level zero.
\end{example}

\begin{example}
    Permutation solutions are multipermutation solutions of level~$1$.
\end{example}

\begin{example}
    et $X=\{1,2,3,4\}$ and let $r:X\times X\to X \times X$ defined by $r(x,y)=(\varphi_x(y),\varphi_y(x))$ where
        \begin{align*}
            \varphi_1=\varphi_2=\operatorname{id}, \quad \varphi_3=(3\ 4), \quad \varphi_4=(1\ 2)(3\ 4).
        \end{align*}
        Then $(X,r)$ is an involutive multipermutation solution of level~$3$.
\end{example}



\begin{proposition}
    Let $B$ be a skew brace and let $(B,r_B)$ be the associated solution. Then, the retraction $\operatorname{Ret}(B,r_B)$ is the solution associated with the quotient skew brace $B/\operatorname{Soc}(B)$.
\end{proposition}


\begin{theorem}\label{thm:ordermultipermutiation}
    Let $(X,r)$ be a finite multipermutation solution to the YBE. If $|X|>1$, then $r$ has even order. 
\end{theorem}


\begin{proof}
    Since $(X,r)\to\Ret(X,r)$, $x\mapsto[x]$ is a homomorphism of solutions, 
    it follows that the order of the solution $\overline{r}$ divides the order of $r$. 
    Assume that $(X,r)$ has multipermutation level $n$. 
    There exists a homomorphism of solutions $(X,r)\to\Ret^{n-1}(X,r)$, thus 
    it is enough to prove the theorem when
    $r(x,y)=(\lambda(y),\rho(x))$ for commuting permutations $\lambda$ and $\rho$, i.e. 
    multipermutation solutions of level $1$. If $r$ has order $2k+1$, then 
    \begin{align*}
        (x,y)=r^{2k+1}(x,y)=(\lambda^{k+1}\rho^k(y),\lambda^k\rho^{k+1}(x)).
    \end{align*}
    This implies that $\lambda^{k+1}\rho^k(y)=x$ for all $x,y\in X$. This equality in particular 
    implies that $x=y$ because $\lambda^{k+1}\rho^k$ is a permutation, a contradiction. 
\end{proof}
% \begin{proof}
%     First of all, let us prove that $\bar{r}$ is well-defined.
%     Let $x,x',y,y'\in X$ such that $x\sim x'$ and $y\sim y'$, i.e. $\lambda_x=\lambda_{x'}$, $\rho_x=\rho_{x'}$, and $\lambda_y=\lambda_{y'}$, $\rho_y=\rho_{y'}$.
%     From Proposition~\ref{prop:characterisation} we have 
%     \begin{align*}
%         \lambda_{\lambda_x(y)}\lambda_{\rho_y(x)}=\lambda_x\lambda_y=\lambda_x\lambda_{y'}
%         =\lambda_{\lambda_x(y')}\lambda_{\rho_{y'}(x)} = \lambda_{\lambda_x(y')}\lambda_{\rho_y(x)}
%     \end{align*}
%     and since $(X,r)$ is non-degenerate we get $\lambda_{\lambda_x(y)}=\lambda_{\lambda_x(y')}=\lambda_{\lambda_{{x'}(y')}}$.
%     Similarly,
%     \begin{align*}
%         \rho_{\rho_y(x)}\rho_{\lambda_x(y)}=\rho_x\rho_y=\rho_x\rho_{y'}
%         =\rho_{\rho_{y'}(x)}\rho_{\lambda_x(y')} = \rho_{\rho_y(x)}\rho_{\lambda_x(y')}
%     \end{align*}
%     and so 
%     $\rho_{\lambda_x(y)}=\rho_{\lambda_x(y')}=\rho_{\lambda_{{x'}(y')}}$. Hence
%     \begin{align*}
%         \lambda_x(y)\sim\lambda_{x'}(y').
%     \end{align*}
%     Following the same procedure we get
%     \begin{align*}
%         \lambda_{\lambda_x(y)}\lambda_{\rho_y(x)}=\lambda_x\lambda_y=\lambda_{x'}\lambda_{y}
%         =\lambda_{\lambda_{x'}(y)}\lambda_{\rho_{y}(x')} = \lambda_{\lambda_x(y)}\lambda_{\rho_{y'}(x')},
%     \end{align*}
%     i.e., $\lambda_{\rho_y(x)}=\lambda_{\rho_{y'}(x')}$. And 
%     \begin{align*}
%         \rho_{\rho_y(x)}\rho_{\lambda_x(y)}=\rho_x\rho_y=\rho_{x'}\rho_{y}
%         =\rho_{\rho_{y}(x')}\rho_{\lambda_x(y)} = \rho_{\rho_{y'}(x')}\rho_{\lambda_x(y)},
%     \end{align*}
%     i.e., $\rho_{\rho_y(x)} = \rho_{\rho_{y'}(x')}$.
%     Hence,
%     \begin{align*}
%         \rho_y(x) \sim \rho_{y'}(x').
%     \end{align*}
%     Therefore, $r$ is well-defined.

%     Now let us prove that $\bar{r}$ is bijective. Since $(X,r)$ is a solution in particular $r$ is bijective. Let us write
%     \begin{align*}
%         r^{-1}(x,y) = (\hat{\lambda}_x(y),\hat{\rho}_y(x)).
%     \end{align*}
%     Since $(X,r^{-1})$ is a solution with the same arguments as before we have that $\overline{r^{-1}}$ is well-defined. Moreover, if $z\in X$, then
%     \begin{align*}
%         \hat{\lambda}^{-1}_x(z) \overset{\eqref{eq:hat1}}{=}\rho_{\lambda^{-1}_x(z)}(x)\sim \rho_{\lambda^{-1}_{x'}(z)}(x)\sim \rho_{\lambda^{-1}_{x'}(z)}(x') \sim \hat{\lambda}^{-1}_{x'}(z)
%     \end{align*}
%     \begin{align*}
%         \overline{r^{-1}}\bar{r}(x,y) =\overline{r^{-1}}([\lambda_x(y)]
%     \end{align*}
% \end{proof}
