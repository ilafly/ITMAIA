\section*{Appendix}

\fancyhf{}
\fancyfoot[R]{\thepage}
\fancyhead[L]{\course}
\fancyhead[R]{Appendix}
\setlength{\headheight}{14pt}



\subsection{Radical rings}
    \begin{definition}
         A non-empty set $R$ with two binary operations the addition~$+$ (addition) and the multiplication~$\cdot$ is a \emph{ring}\index{Ring} if
    \begin{itemize}
        \item $(R,+)$ is an abelian group,
        \item $(R,\cdot)$ is a semigroup (i.e. $\cdot$ is associative),
        \item The multiplication is distributive with respect to the addition, i.e.
        \begin{align*}
            a\cdot (b+c)&=(a\cdot b)+(a\cdot c)&\text{(left distributivity)}\\
            (b+c)\cdot a&=(b\cdot a)+(c\cdot a)&\text{(right distributivity)}
        \end{align*}
        for all $a,b,c\in R$.
    \end{itemize}
    A ring $(R,+,\cdot)$ is \emph{unitary} if there is an element $1$ in $R$ such that $a\cdot 1= 1\cdot a=a$ for all $a \in R$ (i.e., $1$ is the \emph{multiplicative identity}).
    \end{definition}

    Let $R$ be a non-unitary ring. Consider $R_1 = \mathbb{Z}\times R$ with the addition defined component-wise and multiplication
    \begin{align*}
        (k,a)(l,b)=(kl,kb+la+ab)
    \end{align*}
    for all $k,l\in \mathbb{Z}$ and $a,b\in R$.
    
    Then $R_1$ is a ring and $(1, 0)$ is its multiplicative identity.
    
    Note that $\{0\}\times R$ is isomorphic to $R$ as non-unitary rings.

    \begin{exercise}\label{ex:invertible elements}
         Let $R$ be a non-unitary ring. Consider $R_1 = \mathbb{Z}\times R$ as before. If $(k,x)\in R_1$ is invertible, then $k \in\{1,-1\}$.
    \end{exercise}

   \begin{definition}
        Let $R$ be a unitary ring.  The \emph{(Jacobson) radical} $J(R)$ of $R$ is defined as the intersection of all maximal left ideals\footnote{A \emph{left ideal} of $R$ is an additive subgroup $I$ of $R$ such that $ax\in I$ for all $a\in R$ and $x\in I$.} of $R$.
   \end{definition}

   \begin{exercise}\label{ex:Jideal}
       Let $R$ be a unitary ring. 
       \begin{enumerate}
           \item Prove that $J(R)$ in an ideal of $R$.
           \item Prove that  $x\in J(R)$ if and only if $1 + rx$ is invertible for all $r\in R$.
       \end{enumerate}
   \end{exercise}

   \begin{definition}
       A non-unitary ring $R$ is a  \emph{(Jacobson) radical ring} if it is isomorphic
        to the Jacobson radical of a unitary ring. 
   \end{definition}

    \begin{proposition}
        Let $R$ be a non-unitary ring. The following statements are equivalent.
    \begin{enumerate}
        \item $R$ is a radical ring.
        \item For all $a\in R$ there exists a unique $b\in R$ such that $a+b+ab=a+b+ba= 0$.
        \item $R$ is isomorphic to $J(R_1)$.
    \end{enumerate}
    \end{proposition}

    \begin{proof}
        Let us first prove that 1) implies 2). 
        Let $M$ be a unitary ring such that $R$ is isomorphic to its Jacobson radical $J(M)$ and let $\psi: R \to M$ be a homomorphism such that $\psi(R)$ is isomorphic to $J(M)$. 
        Now, if $a\in R$, then $\psi(a)\in J(M)$. By Exercise~\ref{ex:Jideal}, $1+\psi(a)$ is invertible, i.e. there exists $c\in M$ such that 
        \begin{align*}
            (1+\psi(a))(1+c) = 1 = (1+c) (1+\psi(a)).
        \end{align*}
        It follows that $c\in J(M)$, i.e. $c=\psi(b)$ for some $b\in R$. Moreover, since $\psi$ is a homomorphism
        \begin{align*}
            1 &= (1+\psi(a))(1+c) = (1+\psi(a))(1+\psi(b)) \\&= 1 + \psi(a)+\psi(b)+\psi(a)\psi(b)
            = 1 + \psi(a+b +ab)
        \end{align*}
        and 
        \begin{align*}
            1 &= (1+c)(1+\psi(a)) = (1+\psi(b))(1+\psi(a)) \\&= 1 + \psi(b)+\psi(a)+\psi(b)\psi(a)
            = 1 + \psi(a+b+ba).
        \end{align*}
        Hence, 2) holds. 

        Now let us prove 2) implies 3). Let $a\in R$, we aim to prove that $(1,a)\in R_1$ is invertible. By 2) there exists $b\in R$ such that 
        \begin{align*}
            (1,a)(1,b) = (1, a+b+ab) = (1,0)\\
            (1,b)(1,a) = (1, b+a+ba) = (1,0).   
        \end{align*}
        Now, consider $(k,a)\in J(R_1)$. We want to prove that $k=0$, i.e. $J(R_1)\subseteq \{0\}\times R$. Since $(k,a)\in J(R_1)$ follows that $(1,0)+(3,0)(k,a)=)(1+3k,3a)$ is invertible by Exercise~\ref{ex:Jideal}, and so $k=0$. Therefore $J(R_1)\subseteq \{0\}\times R$. Moreover, let $(0,R)\in\{0\}\times R$. then 
        \begin{align*}
            (1,0) + (k,a) (0,x) = (1,0)+(0,kx+ka) = (1, kx+ka)
        \end{align*}
        which is invertible. So $(0,x)\in J(R_1)$.
        Finally the implication 3) implies 1) is trivially true.
    \end{proof}

    \begin{definition}
         Let $R$ be any ring. Define on $R$ the binary operation $\circ$ called the \emph{adjoint multiplication} of $R$
        \begin{align*}
            a\circ b = a+b+ab,
        \end{align*}
        for all $a,b\in R$. 
    \end{definition}

    \begin{lemma}\label{lem:adjoint}
        Then $(R,\circ)$ is a monoid with neutral element $0$. 
    \end{lemma}

    \begin{exercise}
        Prove Lemma~\ref{lem:adjoint}.
    \end{exercise}

    \begin{convention}
        If $a\in R$ is invertible in the monoid $(R, \circ)$, 
        we will denote by $a'$ its inverse.
    \end{convention}

    \begin{examples}\mbox{}
        \begin{enumerate}
            \item Let $p$ be a prime and let $A = \mathbb{Z}/(p^2) = \mathbb{Z}/p^2\mathbb{Z}$ be the ring of integers modulo $p^2$. Then $(A,+)$ with a new multiplication $\ast$ defined by $a\ast b = pab$ is a radical ring. In this case, $a\circ b = a+b+pab$, and $a' =-a+pa^2$. 
            \item Let $n$ be an integer such that $n>1$. Let 
                \begin{align*}
                    A=\left\{\dfrac{nx}{ny+1}\colon x,y \in \mathbb{Z}\right\}\subseteq \mathbb{Q}.
                \end{align*}
                $A$ is a (non-unitary) subring of $\mathbb{Q}$. In fact, $A$ is a radical ring. A straightforward computation shows
                \begin{align*}
                    \left(\dfrac{nx}{ny+1}\right)' = \dfrac{-nx}{n(x+y)+1}.
                \end{align*}
        \end{enumerate}
    \end{examples}