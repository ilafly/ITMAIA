\section{23/02/2024}

\subsection{From solutions to skew braces}

    \begin{definition}
        Let $(X,r)$ be a solution. The \emph{structure group} of $(X,r)$ is the group
        \begin{align*}
            G(X,r)=\gr(X : xy=\lambda_x(y)\rho_y(x) \text{ for all }x,y\in X).
        \end{align*}
        The \emph{derived structure group} of $(X,r)$ is the group
        \begin{align*} 
            A(X,r)=\gr(X : x\lambda_x(y)=\lambda_x(y)\lambda_{\lambda_x(y)}\rho_y(x) \text{ for all }x,y\in X).
        \end{align*}
    \end{definition}

        
    \begin{theorem}[\cite{MR1722951} or \cite{LYZ00}]
    Let $(X,r)$ be a solution.
    Then, there exists a unique structure of skew brace with multiplicative group the structure group isomorphic to $G(X,r)$, additive strucutre isomorphic to $A(X,r)$ and such that $\lambda_{i(x)}(i(y))=i(\lambda_x(y))$ for all $x,y\in X$, where $\iota: X\to G(X,r), x\mapsto x$ is the canoical map. 
    \end{theorem}

    \begin{proof}
        Omitted.
    \end{proof}

    The structure skew brace defined in the previous theorem satisfies the following universal property.
    
    \begin{proposition}
        Let $(B,+,\circ)$ be a skew brace, $j\colon X\rightarrow B$ be a map such that $\lambda_{j(x)}(j(y))=j(\lambda_x(y))$ and $j(x)\circ j(y)=j(\lambda_x(y))\circ j(\rho_y(x))$, for all $x,y\in X$. Then there exists a unique homomorphism of skew braces $f\colon G(X,r)\rightarrow B$ such that $fi=j$, i.e.
        \begin{equation*}
        \begin{tikzcd}
            X \arrow[r,"j"] \arrow[d,swap,"i"] &
            B \arrow[dl,swap, dashed,"f"] 
            \\
            G(X,r) 
        \end{tikzcd}
        \end{equation*}
    \end{proposition}

    \begin{proof}
        Omitted.
    \end{proof}

    \subsection{The permutation group of a solution}

    Let $(X,r)$ be a solution. Consider the structure group $G(X,r)$ of the solution $(X,r)$. Let $i\colon X\to G(X,r)$ be the natural map. 

    \index{Solution!permutation group}
    The \emph{permutation group} of $(X,r)$ is the subgroup
\[\mathcal{G}(X,r)=\langle (\sigma_x,\tau^{-1}_x) : x\in X\rangle\subseteq\Sym_X\times\Sym_X.\]
Since 
\[ \sigma_x\sigma_y=\sigma_{\sigma_x(y)}\sigma_{\tau_y(x)}\quad\text{and}\quad \tau^{-1}_x\tau^{-1}_y=\tau^{-1}_{\sigma_x(y)}\tau^{-1}_{\tau_y(x)} \]
for all $x,y\in X$, there exists a unique group homomorphism
$h\colon G(X,r)\to \mathcal{G}(x,r)$ such that $hi(x)=(\sigma_x,\tau^{-1}_x)$ for all $x\in X$. We write
\[ h(a)=(\sigma_a,\tau_a^{-1})\]
for all $a\in G(X,r)$.
By Theorem \ref{thm:GVbraces1}, $G(X,r)$ has a unique structure of skew brace with multiplicative group the structure group $G(X,r)$ and $\lambda_{i(x)}(i(y))=i(\sigma_x(y))$ for all $x,y\in X$. We shall see that $\ker h$ is an ideal of the skew brace $G(X,r)$. Note that
\[\mu_{i(y)}(i(x))=\lambda_{i(x)}(i(y))^{-1}i(x)i(y)=i(\sigma_x(y))^{-1}i(x)i(y)=i(\tau_y(x))\]
for all $x,y\in X$, by the defining relations of $G(X,r)$.


% \section{The permutation group of a solution.}

% \begin{frame}{The permutation group}
%     Let $(X,r)$ be a solution to the YBE. 
   
%     Write  $r(x,y)=(\lambda_x(y),\rho_y(x))$. 

%     The \alert{permutation group} of $(X,r)$ is the subgroup
%     \[\mathcal{G}(X,r)=\langle (\lambda_x,\rho^{-1}_x) : x\in X\rangle\subseteq\Sym_X\times\Sym_X.\]

% \end{frame}
% \begin{frame}{}
%     From  
%     \[ \lambda_x\lambda_y=\lambda_{\lambda_x(y)}\lambda_{\rho_y(x)}\quad\text{and}\quad \rho^{-1}_x\rho^{-1}_y=\rho^{-1}_{\lambda_x(y)}\rho^{-1}_{\rho_y(x)} \]
%     we get that 

%     there exists a unique group homomorphism
%     $h\colon G(X,r)\to \mathcal{G}(x,r)$ such that $hi(x)=(\lambda_x,\rho^{-1}_x)$ for all $x\in X$.

%     \bigskip
    
    
% \end{frame}

\subsection{The retraction of a solution.}

Let $(X,r)$ be a solution and define on $X$ the following relation
\begin{align*}
    x\sim y \quad \iff \quad \lambda_x=\lambda_y \text{ and } \rho_x=\rho_y.
\end{align*}

Let $\bar{X}=X/\sim$ denote the set of equivalence classes and let $[x]$ denote the class of $x$.

\begin{lemma}\label{lem:hat}
    Let $(X,r)$ be a solution and write $r^{-1}(x,y)= (\hat{\lambda}_x(y),\hat{\rho}_y(x))$.
    Then 
    \begin{align}
        \label{eq:hat1}\hat{\lambda}^{-1}_y(x)=\rho_{\lambda^{-1}_x(y)}(x),\\
        \label{eq:hat2}\lambda^{-1}_x(y)=\hat{\rho}_{\hat{\lambda}^{-1}_y(x)}(y),\\
        \label{eq:hat3}\hat{\rho}^{-1}_x(y)=\lambda_{\rho^{-1}_y(y)}(y),\\
        \label{eq:hat4}\rho^{-1}_y(x) = \hat{\lambda}_{\hat{\rho}^{-1}_x(y)}(x).
    \end{align}
\end{lemma}

\begin{exercise}
    \label{ex:hat}
    Prove Lemma~\ref{lem:hat}.
\end{exercise}

\begin{theorem}
    Let $(X,r)$ be a solution. Then $r$ induce a solution $\bar{r}$ on $\bar{X}$ by
    \begin{align*}
        \bar{r}([x],[y])= ([\lambda_x(y)], [\rho_y(x)]),
    \end{align*}
    for all $x,y\in X$.
\end{theorem}

\begin{proof}
    First of all, let us prove that $\bar{r}$ is well-defined.
    Let $x,x',y,y'\in X$ such that $x\sim x'$ and $y\sim y'$, i.e. $\lambda_x=\lambda_{x'}$, $\rho_x=\rho_{x'}$, and $\lambda_y=\lambda_{y'}$, $\rho_y=\rho_{y'}$.
    From Proposition~\ref{prop:characterisation} we have 
    \begin{align*}
        \lambda_{\lambda_x(y)}\lambda_{\rho_y(x)}=\lambda_x\lambda_y=\lambda_x\lambda_{y'}
        =\lambda_{\lambda_x(y')}\lambda_{\rho_{y'}(x)} = \lambda_{\lambda_x(y')}\lambda_{\rho_y(x)}
    \end{align*}
    and since $(X,r)$ is non-degenerate we get $\lambda_{\lambda_x(y)}=\lambda_{\lambda_x(y')}=\lambda_{\lambda_{{x'}(y')}}$.
    Similarly,
    \begin{align*}
        \rho_{\rho_y(x)}\rho_{\lambda_x(y)}=\rho_x\rho_y=\rho_x\rho_{y'}
        =\rho_{\rho_{y'}(x)}\rho_{\lambda_x(y')} = \rho_{\rho_y(x)}\rho_{\lambda_x(y')}
    \end{align*}
    and so 
    $\rho_{\lambda_x(y)}=\rho_{\lambda_x(y')}=\rho_{\lambda_{{x'}(y')}}$. Hence
    \begin{align*}
        \lambda_x(y)\sim\lambda_{x'}(y').
    \end{align*}
    Following the same procedure we get
    \begin{align*}
        \lambda_{\lambda_x(y)}\lambda_{\rho_y(x)}=\lambda_x\lambda_y=\lambda_{x'}\lambda_{y}
        =\lambda_{\lambda_{x'}(y)}\lambda_{\rho_{y}(x')} = \lambda_{\lambda_x(y)}\lambda_{\rho_{y'}(x')},
    \end{align*}
    i.e., $\lambda_{\rho_y(x)}=\lambda_{\rho_{y'}(x')}$. And 
    \begin{align*}
        \rho_{\rho_y(x)}\rho_{\lambda_x(y)}=\rho_x\rho_y=\rho_{x'}\rho_{y}
        =\rho_{\rho_{y}(x')}\rho_{\lambda_x(y)} = \rho_{\rho_{y'}(x')}\rho_{\lambda_x(y)},
    \end{align*}
    i.e., $\rho_{\rho_y(x)} = \rho_{\rho_{y'}(x')}$.
    Hence,
    \begin{align*}
        \rho_y(x) \sim \rho_{y'}(x').
    \end{align*}
    Therefore, $r$ is well-defined.

    Now let us prove that $\bar{r}$ is bijective. Since $(X,r)$ is a solution in particular $r$ is bijective. Let us write
    \begin{align*}
        r^{-1}(x,y) = (\hat{\lambda}_x(y),\hat{\rho}_y(x)).
    \end{align*}
    Since $(X,r^{-1})$ is a solution with the same arguments as before we have that $\overline{r^{-1}}$ is well-defined. Moreover, if $z\in X$, then
    \begin{align*}
        \hat{\lambda}^{-1}_x(z) \overset{\eqref{eq:hat1}}{=}\rho_{\lambda^{-1}_x(z)}(x)\sim \rho_{\lambda^{-1}_{x'}(z)}(x)\sim \rho_{\lambda^{-1}_{x'}(z)}(x') \sim \hat{\lambda}^{-1}_{x'}(z)
    \end{align*}
    \begin{align*}
        \overline{r^{-1}}\bar{r}(x,y) =\overline{r^{-1}}([\lambda_x(y)]
    \end{align*}
\end{proof}



% \begin{frame}{}
%     $\bar{r}$ is bijective. 
%     \vspace{6cm}
% \end{frame}


% \begin{frame}{}
%     $\bar{r}$ is non-degenerate. 
%     \vspace{6cm}
% \end{frame}

% \begin{frame}{}
    
% \end{frame}

% \begin{frame}{}
    
% \end{frame}

% \begin{frame}{Definition}
%     Let $(X,r)$ be a solution. The solution $\operatorname{Ret}(X,r) =(\bar{X},\bar{r})$ induced by the equivalence relation $\sim$ is the \alert{retraction} of $(X,r)$.

%     \bigskip

%     We define inductively $\operatorname{Ret}^{0}(X,r)=(X,r)$, $\operatorname{Ret}^1(X,r)=\operatorname{Ret}(X,r)$ and 
%     \begin{align*}
%         \operatorname{Ret}^{n+1}(X,r) = \operatorname{Ret}(\operatorname{Ret}^n(X,r)), \quad n\geq 1.
%     \end{align*}
% \end{frame}

% \subsection{Multipermutation and irretractable solutions.}
% \begin{frame}{Multipermutation}
%     A solution $(X,r)$ is said to be a \alert{multipermutation solution of level} $n$, if $n$ is the smallest non-negative integer such that $|\operatorname{Ret}^{n}(X,r)| = 1$. 
    
%     \bigskip
    
%     The solution $(X,r)$ is said to be \alert{irretractable} if $\operatorname{Ret}(X,r)=(X,r)$. 
% \end{frame}


% \begin{frame}{Examples}
%     \begin{itemize}
%         \item The trivial solution over a set with one element is a multipermutation solution of level zero.
%         \item[]
%         \item Permutation solutions are multipermutation solutions of level~$1$.
%         \item[] 
%         \item Let $X=\{1,2,3,4\}$ and let $r:X\times X\to X \times X$ defined by $r(x,y)=(\varphi_x(y),\varphi_y(x))$ where
%         \begin{align*}
%             \varphi_1=\varphi_2=\operatorname{id}, \quad \varphi_3=(3\ 4), \quad \varphi_4=(1\ 2)(3\ 4).
%         \end{align*}
%         Then $(X,r)$ is an involutive multipermutation solution of level~$3$.
%     \end{itemize}
% \end{frame}


% \begin{frame}{}
%     \textbf{Theorem. } Let $B$ be a skew brace and let $(B,r_B)$ be the associated solution. Then, the retraction $\operatorname{Ret}(B,r_B)$ is the solution associated with the quotient skew brace $B/\operatorname{Soc}(B)$.
% \end{frame}

% \begin{frame}{}
    
% \end{frame}

% \begin{frame}{}
    
% \end{frame}

\begin{theorem}\label{thm:ordermultipermutiation}
    Let $(X,r)$ be a finite multipermutation solution to the YBE. If $|X|>1$, then $r$ has even order. 
\end{theorem}

\begin{proof}
    Since $(X,r)\to\Ret(X,r)$, $x\mapsto[x]$ is a homomorphism of solutions, 
    it follows that the order of the solution $\overline{r}$ divides the order of $r$. 
    Assume that $(X,r)$ has multipermutation level $n$. 
    There exists a homomorphism of solutions $(X,r)\to\Ret^{n-1}(X,r)$, thus 
    it is enough to prove the theorem when
    $r(x,y)=(\lambda(y),\rho(x))$ for commuting permutations $\lambda$ and $\rho$, i.e. 
    multipermutation solutions of level $1$. If $r$ has order $2k+1$, then 
    \begin{align*}
        (x,y)=r^{2k+1}(x,y)=(\lambda^{k+1}\rho^k(y),\lambda^k\rho^{k+1}(x)).
    \end{align*}
    This implies that $\lambda^{k+1}\rho^k(y)=x$ for all $x,y\in X$. This equality in particular 
    implies that $x=y$ because $\lambda^{k+1}\rho^k$ is a permutation, a contradiction. 
\end{proof}