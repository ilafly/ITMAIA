\section{23/02/2024}

\subsection{From solutions to skew braces}

    \begin{definition}
        Let $(X,r)$ be a solution. The \emph{structure group} of $(X,r)$ is the group
        \begin{align*}
            G(X,r)=\langle X : xy=\lambda_x(y)\rho_y(x) \text{ for all }x,y\in X\rangle.
        \end{align*}
        The \emph{derived structure group} of $(X,r)$ is the group
        \begin{align*} 
            A(X,r)=\langle X : x\lambda_x(y)=\lambda_x(y)\lambda_{\lambda_x(y)}\rho_y(x) \text{ for all }x,y\in X\rangle.
        \end{align*}
    \end{definition}

        
    \begin{theorem}\label{thm:structuregroup}[\cite{MR1722951} or \cite{LYZ00}]
    Let $(X,r)$ be a solution.
    Then, there exists a unique structure of skew brace with multiplicative group the structure group isomorphic to $G(X,r)$, additive structure isomorphic to $A(X,r)$ and such that $\lambda_{i(x)}(i(y))=i(\lambda_x(y))$ for all $x,y\in X$, where $i: X\to G(X,r), x\mapsto x$ is the canonical map. 
    \end{theorem}

    \begin{proof}
        Omitted.
    \end{proof}

    The structure skew brace defined in the previous theorem satisfies the following universal property.
    
    \begin{proposition}
        Let $(B,+,\circ)$ be a skew brace, $j\colon X\rightarrow B$ be a map such that $\lambda_{j(x)}(j(y))=j(\lambda_x(y))$ and $j(x)\circ j(y)=j(\lambda_x(y))\circ j(\rho_y(x))$, for all $x,y\in X$. Then there exists a unique homomorphism of skew braces $f\colon G(X,r)\rightarrow B$ such that $fi=j$, i.e.
        \begin{equation*}
        \begin{tikzcd}
            X \arrow[r,"j"] \arrow[d,swap,"i"] &
            B \arrow[dl,swap, dashed,"f"] 
            \\
            G(X,r) 
        \end{tikzcd}
        \end{equation*}
    \end{proposition}

    \begin{proof}
        Omitted.
    \end{proof}

    \subsection{The permutation group of a solution}

    Let $(X,r)$ be a solution. Consider the structure group $G(X,r)$ of the solution $(X,r)$. Let $i\colon X\to G(X,r)$ be the natural map. 

    The \emph{permutation group} of $(X,r)$ is the subgroup
    \[\mathcal{G}(X,r)=\langle (\lambda_x,\sigma^{-1}_x) : x\in X\rangle\subseteq\Sym_X\times\Sym_X.\]
    
    Since 
    \begin{align*}
        \lambda_x\lambda_y=\lambda_{\lambda_x(y)}\lambda_{\rho_y(x)}\quad\text{and}\quad \rho^{-1}_x\rho^{-1}_y=\rho^{-1}_{\lambda_x(y)}\rho^{-1}_{\rho_y(x)}
    \end{align*}
    for all $x,y\in X$, there exists a unique group homomorphism
    $h\colon G(X,r)\to \mathcal{G}(x,r)$ such that $hi(x)=(\lambda_x,\rho^{-1}_x)$ for all $x\in X$. We write
    \[ h(a)=(\lambda_a,\rho_a^{-1})\]
    for all $a\in G(X,r)$.
    By Theorem~\ref{thm:structuregroup}, $G(X,r)$ has a unique structure of skew brace with multiplicative group the structure group $G(X,r)$ and $\lambda_{i(x)}(i(y))=i(\lambda_x(y))$ for all $x,y\in X$. One can prove that $\ker h$ is an ideal of the skew brace $G(X,r)$. This allows to give a skew brace structure to the permutation group.

\subsection{Simple solutions}

Describing all solutions is a very difficult task. A strategy to tackle such a problem is to focus on ``building blocks'' such as indecomposable and simple solutions. 

\begin{definition}
    A solution $(X,r)$ is said to be \emph{decomposable} if there exist $\emptyset \neq Y,Z \subseteq X$ such that $Y\cup Z = X$, $Y \cap Z = \emptyset$ and $r(Y\times Y)\subseteq Y\times Y$, $r(Z\times Z)\subseteq Z \times Z$. Otherwise, $(X,r)$ is said \emph{indecomposable}.
\end{definition}

It is not difficult to prove that a solution $(X,r)$ is indecomposable if the group $\langle \lambda_x, \rho_y \colon x, y \in \rangle$ acts transitively on $X$.

\begin{definition}
    A solution $(X,r)$ is said to be \emph{simple} if for any epimorphism of solutions $\varphi: (X,r) \to (Y,t)$, we have either $\varphi$ an isomorphism or $Y$ a singleton.
\end{definition}

One can prove that a simple solution is, in particular, indecomposable. 

Cedó and Okniński in \cite{MR4391683} proved that some simple skew braces with additive structure abelian provide examples of (involutive) simple solutions. 

Joyce, in 1982, studied simple racks and provided an algebraic characterisation of such racks. By Exercise~\ref{ex2}, a rack provides a solution, and it is not difficult to see that simple racks are, in particular, examples of simple solutions. 

In joint work with Jespers, Kubat, and Van Antwerpen \cite{colazzo2023simple}, we obtain a brace theoretic classification of simple solutions. 

\begin{theorem}
    A simple solution $(X,r)$ is one of the following type:
    \begin{enumerate}
        \item $(X,r)$ is a simple permutational solution,
        \item $(X,r)$ is a simple square-free\footnote{A solution $(X,r)$ is square-free if $r(x,x)=(x,x)$ for every $x\in X$.} derived solution (i.e. it is a simple quandle\footnote{A rack $(X,\triangleleft)$ is a quandle if $x\triangleleft x = x$, for every $x\in X$.}),
        \item $(X,r)$ is a simple solution embedded in a solution associated with a finite skew brace. 
    \end{enumerate}
\end{theorem}

\subsubsection{Simple permutational solution}

Recall that a $(X,r)$ is a permutational solution if there exist $\lambda,\rho$ commuting permutations such that $r(x,y)=(\lambda(y),\rho(x))$.

We can give a combinatorial characterisation of simple permutational solutions.

\begin{proposition}
    A permutational solution $(X,r)$ is simple if and only if the cardinality of $X$ is a prime number $p$, the group $H=\langle\lambda,\rho\rangle$ is cyclic of order $p$.

    In particular, $(X,r)$ solution is isomorphic to $(\mathbb{Z}_p,t)$ such that $t(x,y)= (y+a,x+b)$ with $,a,b\in \mathbb{Z}$ and$(a,b)\neq (0,0)$.
\end{proposition}

\subsubsection{Simple quandle}

The description of simple square-free derived solutions coincides with the description obtained by Joyce of simple quandles. 

\begin{proposition}
    A square-free derived solution $(X,r)$ is simple if and only if $(X,r)$ embeds in a solution $(A,r)$ where $A$ is a finite group and $r(y, y^{-1}xy)$ such that
    \begin{itemize}
        \item $X$ is a conjucacy class generating $A$,
        \item the derived subgroup $[A,A]$ is the smallest non-zero normal subgroup of $A$,
        \item the quotient group $A/[A,A]$ is a cyclic group,
        \item the center $Z(A)$ is trivial.
    \end{itemize}
\end{proposition}

\subsubsection{Simple solutions embedded in solutions associated with finite skew braces}

Let $(A,+,\circ)$ be a skew brace. In analogy with radical rings, we can associate to $A$ another binary operation $\ast:A\times A\to A$ such that $a\ast b = -a+a\circ b +b$, for every $a,b\in A$.

\begin{exercise}\label{ex:A2}
    Let $(A,+,\circ)$ be a skew brace. Define $A^2$ the additive subgroup generated by $\{a\ast b=-a+a\circ b - b \colon a,b\in A\}$. Prove that 
    $A^2$ is an ideal of $A$.
\end{exercise}

The ideal $A^2$ plays for skew braces, a similar role as that the derived subgroup plays for groups. With this observation in mind, the following proposition might not be surprising. 

\begin{proposition}
    A square-free solution $(X,r)$, which is not a derived solution nor a permutational solution, is simple if and only if $(X,r)$ is embedded in a solution associated with a non-trivial skew brace $(A,r_A)$ such that
    \begin{itemize}
        \item $X$ generates additively $A$,
        \item the ideal $A^2$ is the smallest non-zero ideal of $A$,
        \item the quotient skew brace $A/A^2$ is a trivial skew brace with additive (and multiplicative) group cyclic,
        \item the action of $A^2$ on $X$ is transitive.
    \end{itemize}
\end{proposition}

Let us explain what the last item means. Let $A$ be a skew brace, $X\subset A$ and $I$ an ideal of $A$. We say that $I$ \emph{acts} on $X$ if $X$ is invariant under the action of the group $(I,+)\rtimes (I,\circ)$ (where the $\ lambda$-map gives the semidirect product).