\section{23/02/2024}

\subsection{From solutions to skew braces}

    \begin{definition}
        Let $(X,r)$ be a solution. The \emph{structure group} of $(X,r)$ is the group
        \begin{align*}
            G(X,r)=\gr(X : xy=\lambda_x(y)\rho_y(x) \text{ for all }x,y\in X).
        \end{align*}
        The \emph{derived structure group} of $(X,r)$ is the group
        \begin{align*} 
            A(X,r)=\gr(X : x\lambda_x(y)=\lambda_x(y)\lambda_{\lambda_x(y)}\rho_y(x) \text{ for all }x,y\in X).
        \end{align*}
    \end{definition}

        
    \begin{theorem}\label{thm:structuregroup}[\cite{MR1722951} or \cite{LYZ00}]
    Let $(X,r)$ be a solution.
    Then, there exists a unique structure of skew brace with multiplicative group the structure group isomorphic to $G(X,r)$, additive structure isomorphic to $A(X,r)$ and such that $\lambda_{i(x)}(i(y))=i(\lambda_x(y))$ for all $x,y\in X$, where $i: X\to G(X,r), x\mapsto x$ is the canonical map. 
    \end{theorem}

    \begin{proof}
        Omitted.
    \end{proof}

    The structure skew brace defined in the previous theorem satisfies the following universal property.
    
    \begin{proposition}
        Let $(B,+,\circ)$ be a skew brace, $j\colon X\rightarrow B$ be a map such that $\lambda_{j(x)}(j(y))=j(\lambda_x(y))$ and $j(x)\circ j(y)=j(\lambda_x(y))\circ j(\rho_y(x))$, for all $x,y\in X$. Then there exists a unique homomorphism of skew braces $f\colon G(X,r)\rightarrow B$ such that $fi=j$, i.e.
        \begin{equation*}
        \begin{tikzcd}
            X \arrow[r,"j"] \arrow[d,swap,"i"] &
            B \arrow[dl,swap, dashed,"f"] 
            \\
            G(X,r) 
        \end{tikzcd}
        \end{equation*}
    \end{proposition}

    \begin{proof}
        Omitted.
    \end{proof}

    \subsection{The permutation group of a solution}

    Let $(X,r)$ be a solution. Consider the structure group $G(X,r)$ of the solution $(X,r)$. Let $i\colon X\to G(X,r)$ be the natural map. 

    \index{Solution!permutation group}
    The \emph{permutation group} of $(X,r)$ is the subgroup
    \[\mathcal{G}(X,r)=\langle (\lambda_x,\sigma^{-1}_x) : x\in X\rangle\subseteq\Sym_X\times\Sym_X.\]
    
    Since 
    \begin{align*}
        \lambda_x\lambda_y=\lambda_{\lambda_x(y)}\lambda_{\rho_y(x)}\quad\text{and}\quad \rho^{-1}_x\rho^{-1}_y=\rho^{-1}_{\lambda_x(y)}\rho^{-1}_{\rho_y(x)}
    \end{align*}
    for all $x,y\in X$, there exists a unique group homomorphism
    $h\colon G(X,r)\to \mathcal{G}(x,r)$ such that $hi(x)=(\lambda_x,\rho^{-1}_x)$ for all $x\in X$. We write
    \[ h(a)=(\lambda_a,\rho_a^{-1})\]
    for all $a\in G(X,r)$.
    By Theorem~\ref{thm:structuregroup}, $G(X,r)$ has a unique structure of skew brace with multiplicative group the structure group $G(X,r)$ and $\lambda_{i(x)}(i(y))=i(\lambda_x(y))$ for all $x,y\in X$. One can prove that $\ker h$ is an ideal of the skew brace $G(X,r)$. This allows to give a skew brace strucure to the permution group.

\subsection{Simple solutions and skew braces}

A characterisation of arbitrary solutions is presently beyond reach. Hence one needs to
restrict the investigations to building blocks of such solutions. 

Of course a first and obvious restriction is to deal with
indecomposable solutions and with simple solutions. 

\begin{definition}
    A solution $(X,r)$  is said to be \emph{decomposable} if $X=Y\cup Z$, a disjoint
    union of non-empty subsets, and $r$ restricted to $Y\times Y$ and $Z\times Z$ respectively defines a solution on $Y$ and $Z$. It is said \emph{indecomposable} otherwise.
\end{definition}

\begin{definition}
    A solution $(X,r)$ is said to be \emph{simple} if any epimorphic image is either isomorphic to $(X,r)$ or to a solution $(Y,s)$ with $Y$ a singleton. 
\end{definition}


    Okni\'nski and Ced\'o \cite{CeOkn} proved that some finite simple skew  braces of abelian type determine an involutive
    simple solution $(X,r)$. They showed that finite simple braces which are additively generated by an orbit $X$ under the action
    of the permutation skew  brace $\mathcal{G}(X,r)$ yield simple non-degenerate involutive solutions on $X$. This connection
    between involutive simple solutions and simple skew  braces of abelian type recently has been refined by Castelli in
    \cite{Cast22} by showing that the simplicity of a finite non-degenerate involutive solution $(X,r)$ is described by the
    algebraic structure of its permutation skew brace.

    In \cite{simple} we provide a brace-theoretical classification of simple solutions.
    We can summarise the main theorem as follows.
    
    \begin{theorem}
        A finite solution $(X,r)$ of the Yang--Baxter equation is simple if and only if
    it is one of the following types: 
    \begin{enumerate}
        \item $(X,r)$ is a simple permutation solution.
        \item $(X,r)$ is a simple derived solution.
        \item $(X,r)$ is a simple subsolution of a solution associated with a skew  brace $B$.
    \end{enumerate}
    \end{theorem}

    And we can further describe solutions of each type. 

    \subsection{simple permutation solutions}

    % \begin{theorem}
    %     Assume $(X,r)$ is a finite permutation solution with $|X|>1$, i.e., $r(x,y)=(\lambda(y),\rho(x))$
    % for some commuting permutations $\lambda,\rho\in\Sym(X)$. Then $(X,r)$ is simple if and only if $|X|=p$
    % for some prime $p$ and the group $H=\free{\lambda,\rho}$ is cyclic of order $p$. In other words, identifying $X$
    % with $\mb{Z}_{p}$, such solutions are isomorphic with solutions of the form
    % $\mb{Z}_{p}^2\to\mb{Z}_p^{2}\colon(x,y)\mapsto(x+a,y+b) $ with $a,b\in \mb{Z}_{p}$ and $(a,b)\ne (0,0)$.
    % Actually we can take $a=1$ or $b=1$.
    % \end{theorem}
    

    \begin{theorem}
        A finite solution $(X,r)$ of the Yang--Baxter equation is simple if and only if
    it is one of the following types: 
    \begin{enumerate}
        \item $(X,r)$ is a simple permutation solution.
        \item $(X,r)$ is a derived solution that is embedded into a solution $(B,r)$ defined by $r(x,y)=(y, y^{-1}xy)$ with $B$ a finite group satisfying 
        the following properties:
        \begin{itemize}
            \item $X$ is a conjugacy class that generates $B$,
            \item the derived subgroup $[B,B]$ is the smallest non-zero normal subgroup of $B$,
            \item $B/[B,B]$ is a cyclic group,
            \item the center $Z(B)$ of $B$ is trivial.
        \end{itemize}
        \item $(X,r)$ is a subsolution of a solution associated with a skew  brace $B$ such that 
        \begin{itemize}
            \item $X$ generates $(B,+)$,
            \item the ideal $B^2$ is the smallest non-zero ideal of $B$,
            \item $B/B^2$ is a trivial skew left brace of cyclic type,
            \item the action of the ideal $B^2$ on $X$ is transitive.
        \end{itemize}
    \end{enumerate}
    \end{theorem}