\usepackage[foot]{amsaddr} 
\usepackage{hyperref}
\usepackage{listings}
\usepackage{tikz-cd}
\usepackage{datetime}
\usepackage{mathptmx}
\usepackage{amssymb}
\usepackage{newtxtext}
\usepackage{mathtools}
\usepackage{stmaryrd}
\usepackage{mdframed}
\usepackage{textcomp}

\usepackage[margin=1in,footskip=.25in]{geometry}

\overfullrule=1mm

\lstdefinelanguage{Julia}%
  {morekeywords={abstract,break,case,catch,const,continue,do,else,elseif,%
      end,export,false,for,function,immutable,import,importall,if,in,%
      macro,module,otherwise,quote,return,switch,true,try,type,typealias,%
      using,while},%
   sensitive=true,%
   alsoother={$},%
   morecomment=[l]\#,%
   morecomment=[n]{\#=}{=\#},%
   morestring=[s]{"}{"},%
   morestring=[m]{'}{'},%
}[keywords,comments,strings]%

\definecolor{background}{HTML}{F5F5F5}
\definecolor{jlstring}{HTML}{880000}%          % julia's strings
\definecolor{jlbase}{HTML}{444444}%            % julia's base color
\definecolor{jlkeyword}{HTML}{444444}%         % julia's keywords
\definecolor{jlliteral}{HTML}{78A960}%         % julia's literals
\definecolor{jlbuiltin}{HTML}{397300}%         % julia's built-ins
\definecolor{jlmacros}{HTML}{1F7199}%          % julia's macros
\definecolor{jlfunctions}{HTML}{444444}%       % julia's functions
\definecolor{jlcomment}{HTML}{888888}%         % julia's comments
\definecolor{jlstring}{HTML}{880000}%          % julia's strings


\lstset{%
    language         = Julia,
    basicstyle       = \color{jlstring}\ttfamily\scriptsize,
    backgroundcolor  = \color{background},
    keywordstyle     = \color{jlkeyword},
    stringstyle      = \color{jlstring},
    commentstyle     = \color{jlcomment},
    showstringspaces = false,
    columns=fixed,
}

\renewcommand\sectionname{Lecture}
\renewcommand\subsectionname{\S}

% para enumerar
\renewcommand{\labelenumi}{\textbf{\arabic{enumi})}}

\usepackage[most]{tcolorbox}

\newtcolorbox{mybox}{
enhanced,
boxrule=0pt,frame hidden,
%borderline west={4pt}{0pt}{black},
colback={gray!20},
sharp corners
}

\makeindex             


\newcommand{\Hom}{\operatorname{Hom}}
\newcommand{\id}{\operatorname{id}}
\newcommand{\Aut}{\operatorname{Aut}}
\newcommand{\Inn}{\operatorname{Inn}}
\newcommand{\End}{\operatorname{End}}
\newcommand{\Ret}{\operatorname{Ret}}
\newcommand{\Soc}{\operatorname{Soc}}
\newcommand{\Fix}{\operatorname{Fix}}
\newcommand{\Sym}{\operatorname{Sym}}
\newcommand{\Ann}{\operatorname{Ann}}
\newcommand{\gr}{\operatorname{gr}}






\newtheorem{theorem}{Theorem}[section]
\newtheorem{lemma}[theorem]{Lemma}
\newtheorem{proposition}[theorem]{Proposition}
\newtheorem{corollary}[theorem]{Corollary}

\theoremstyle{definition}
\newtheorem{definition}[theorem]{Definition}
\newtheorem{convention}{Convention}
\newtheorem{example}[theorem]{Example}
\newtheorem{examples}[theorem]{Examples}
\newtheorem{xca}[theorem]{Exercise}
%\newmdtheoremenv{exercise}[theorem]{Exercise}
\newtheorem{remark}[theorem]{Remark}

\theoremstyle{remark}
\newtheorem*{claim}{Claim}

\newenvironment{sol}[1]
{\renewcommand{\qedsymbol}{}\begin{proof}[\ref{#1}]}
  {\end{proof}}

\newenvironment{exercise}
  {\begin{mybox}\begin{xca}}
  {\end{xca}\end{mybox}}
  
\numberwithin{equation}{section}

\makeindex

\title{\course}
\author{Ilaria Colazzo}
\address{University of Exeter -- Exeter (UK)}
\email{ilariacolazzo@gmail.com}
\thanks{}
\date{}

\makeatletter

\usepackage{fancyhdr}
\pagestyle{fancy}
\fancyhf{}
\fancyfoot[R]{\thepage}
\fancyhead[L]{\course}
\fancyhead[R]{Lecture \thesection}
\setlength{\headheight}{14pt}


\usepackage{tikz}
\usetikzlibrary{braids}
\usetikzlibrary{cd}



\usepackage{imakeidx}
\makeindex[program=makeindex,columns=2,intoc=true,options={-s index_style.ist}]
