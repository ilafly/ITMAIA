\section{21/02/2024}

\subsection{The Yang--Baxter equation}
The Yang-Baxter equation (YBE) is one important equation in mathematical physics.
It first appeared in two independent papers of Yang \cite{Yang1967} and Baxter\
\cite{Baxter1971}.

\begin{definition}\index{Yang--Baxter equation}
    A solution of the \emph{Yang--Baxter eqution} is a linear map  $R: V\otimes V \to V \otimes V$, where $V$ is a vector space such that
    \begin{align*}
        R_{12}R_{13}R_{23} = R_{23}R_{13}R_{12}
    \end{align*}
    where $R_{ij}$ denotes the map $V\otimes V\otimes V \to V \otimes V \otimes V$ acting as $R$ on the $(i,j)$
    factor and as the identity on the remaining factor.
\end{definition}
    Let $\tau: V\otimes V \to V \otimes V$ be the map $\tau(u\otimes v) = v \otimes u$ for $u,v \in V$.
    It's easy to check (try!) that $R: V \otimes V \to V \otimes V$ is a solution of the Yang--Baxter equation if and only if $\bar{R}:=\tau R$ satisfies 
    \begin{align*}
        \bar{R}_{12}\bar{R}_{23}\bar{R}_{12}=\bar{R}_{23}\bar{R}_{12}\bar{R}_{23}.
    \end{align*}

    An interesting class of solutions of the Yang--Baxter equation arises when $V$ has a $R$-invariant basis $X$. In such a case the solution is said to be set-theoretic. 

\subsection{The set-theoretic version}
    Drinfeld in \cite{Dri1992} observed it makes sense to consider the Yang--Baxter equation in the category of sets and stated that 

    
    \begin{quote}\textit{it would be interesting to study set-theoretic solutions.}\end{quote}
    

    These lectures will focus on set-theoretic solutions to the Yang-Baxter equation and their connection with known and ``new'' algebraic structures. 

    \begin{definition}\index{Set-theoretic solution}
        A \emph{set-theoretic solution to the Yang--Baxter equation} is a pair $(X,r)$ where $X$ is a non-empty set and $r: X\times X \to X \times X$ is a map such that
        \begin{align}\label{eq:YBE}
            (r\times \id)(\id \times r)(r\times \id) = (\id \times r)(r\times \id)(\id \times r)
        \end{align}
    \end{definition}
    
    \begin{convention}
        If $(X,r)$ is a set-theoretic solution to the Yang--Baxter equation, we write 
        \begin{align*}
            r(x,y) = (\lambda_x(y),\rho_y(x))        
        \end{align*}
        where $\lambda_x,\rho_x:X\to X$.
    \end{convention}


    \begin{definition}\index{Set-theoretic solution!Finite}\index{Set-theoretic solution!Non-degenerate}
        Let $(X,r)$ be a set-theoretic solution to the  Yang--Baxter equation. We say that 
        \begin{itemize}
            \item $(X,r)$ is \emph{bijective} if $r$ is bijective.
            \item $(X,r)$ is \emph{finite} if $X$ is finite.
            \item $(X,r)$ is \emph{non-degenerate} if $\lambda_x,\rho_x$ are bijective for all $x\in X$.
        \end{itemize}
    \end{definition}



\subsection{A characterisation}
    \begin{proposition}\label{prop:characterisation}
        Let $X$ be a non-empty set and $r: X \times X \to X \times X$ be a  map, written ase $r(x,y) = (\lambda_x(y),\rho_y(x))$. Then $r$ satisfies equation~\ref{eq:YBE} if and only if
    \begin{enumerate}
        \item $\lambda_x\lambda_y = \lambda_{\lambda_x(y)}\lambda_{\rho_y(x)}$
        \item $\lambda_{\rho_{\lambda_y(z)}(x)}\rho_z(y)=\rho_{\lambda_{\rho_y(x)}(z)}\lambda_x(y)$
        \item $\rho_z\rho_y=\rho_{\rho_z(y)}\rho_{\lambda_y(z)}$
    \end{enumerate}
    for all $x,y,z\in X$.

    In particular, $(X,r)$ is a solution to the Yang--Baxter equation when $r$ is bijective.
    \end{proposition}
    \begin{proof}
        Let us write $r_1=r\times \id$ and $r_2=\id \times r$. Then  
        \begin{align*}
            r_1r_2r_1(x,y,z)& = r_1r_2(\lambda_x(y),\rho_y(x),z) \\
            &= r_1(\lambda_x(y),\lambda_{\rho_y(x)}(z),\rho_z\rho_y(x))\\
            &= (\lambda_{\lambda_x(y)}\lambda_{\rho_y(x)}(z),\rho_{\lambda_{\rho_y(x)}(z)}\lambda_x(y),\rho_z\rho_y(x)),
        \end{align*}
        and 
        \begin{align*}
            r_2r_1r_2(x,y,z) =& r_2r_1(x,\lambda_y(z),\rho_z(y)) \\
            &= r_2(\lambda_x\lambda_y(z),\rho_{\lambda_y(z)}(x),\rho_z(y))\\
            &= (\lambda_x\lambda_y(z),\lambda_{\rho_{\lambda_y(z)}(x)}\rho_z(y),\rho_{\rho_z(y)}\rho_{\lambda_y(z)}(x)).
        \end{align*}
        Therefore $r_1r_2r_1= r_2r_1r_2$ if and only if $1), 2)$ and $3)$ hold.     
    \end{proof}

\subsection{First examples}
    \begin{examples}
        Let $X$ be a non-empty set.  
        \begin{enumerate}
            \item The pair $(X,\id_{X\times X})$ is a  set-theoretic solution to the  Yang--Baxter equation. Note that $(X,\id_{X\times X})$ is not non-degenerate, since $\lambda_x(y)=x$ and $\rho_y(x)=y$, for all $x,y\in X$.
            \item Let $\tau: X \times X \to X \times X$ be the flip map, i.e. $\tau(x,y)=(y,x)$ for all $x,y \in X$. Then, the pair $(X,\tau)$ is a set-theoretic solution to the Yang--Baxter equation. Moreover, it is non-degenerate since $\lambda_x=\rho_x=\id_X$ for all $x\in X$.
            \item Let $\lambda, \rho$ be permutations of $X$. Then $r(x,y)=(\lambda(y),\rho(x))$ is a
            non-degenerate set-theoretic solution to the Yang--Baxter equation if and only if $\lambda\rho = \rho\lambda$.
            Moreover, $(X , r )$ is involutive if and only if $\rho = \lambda^{-1}$. The solution 
            $(X,r)$ is called a \emph{permutational solution} or a \emph{Lyubashenko's solution}.
        \end{enumerate}
    \end{examples}

    If, on the set $X$, we have a bit more structure, we can define some more sophisticated solutions.
    \begin{example}
        Let $G$ be a group and, let
        \begin{align*}
            r_1(x,y) &= (y, y^{-1}xy)\\
            r_2(x,y) &= (x^{2}y, y^{-1}x^{-1}y).
        \end{align*}
        Then $(X,r_1)$ and $(X,r_2)$ are bijective non-degenerate set-theoretic solutions to the Yang--Baxter equation.
    \end{example}

\subsection{Set-theoretic solutions to the Yang--Baxter equation and III Reidemeister move}
    Let us represent the map $r:X \times X \to X \times X$ as a crossing and the identity on $X$ as a straight line; see Figure~\ref{fig:crossing}. \index{III Reidemeister move}
    \begin{figure}[h!]
        \centering
        \begin{tikzpicture}
            \pic[braid/.cd,
                number of strands=3,
                ultra thick,
                gap=0.1,
                name prefix=braid,
            ] {braid={a_1}};
        \end{tikzpicture}
        \caption{The map $r$ represented by a crossing and the identity as a straight line.}
        \label{fig:crossing}
    \end{figure}

    Then the Yang--Baxter equation can be pictured as in Figure~\ref{fig:YangBaxter}.

    \begin{figure}
        \centering
        \begin{tikzpicture}
            \pic[braid/.cd,
                number of strands=3,
                ultra thick,
                gap=0.1,
                name prefix=braid,
            ] {braid={a_{1}a_2a_1}};
            \node[font=\Large] at (3,-2) {\(=\)};
        \end{tikzpicture}
        \hspace{.5cm}
        \begin{tikzpicture}
        \pic[braid/.cd,
            number of strands=3,
            ultra thick,
            gap=0.1,
            name prefix=braid,
        ] {braid={a_{2}a_1a_2}};
        \end{tikzpicture}
        \caption{The Yang--Baxter equation.}
        \label{fig:YangBaxter}
    \end{figure}

    Moreover, we have the following lemma under the assumption of $(X,r)$ being non-degenerate 
    
    \begin{lemma}
    \label{lem:LYZ}
        Let $(X,r)$ be a solution to the Yang--Baxter equation. 
        \begin{enumerate}
            \item Given $x,u\in X$, there exist unique $y,v\in X$ such that $r(x,y)=(u,v)$. 
            \item Given $y,v\in X$, there exist unique $x,u\in X$ such that $r(x,y)=(u,v)$. 
        \end{enumerate}
    \end{lemma}
    \begin{proof}
        For the first claim take $y=\lambda_x^{-1}(u)$ and $v=\rho_y(x)$. 
        For the second, $x=\rho_y^{-1}(v)$ and $u=\lambda_x(y)$. 
    \end{proof}

    So, the bijectivity of $r$ means that any row in Figure~\ref{fig:braid} determines the whole square. By Lemma~\ref{lem:LYZ} we have that non-degeneracy means that any column in Figure~\ref{fig:braid} also determines the entire square. 

    \begin{figure}[h!]
    \centering
        \begin{tikzpicture}
            \pic[braid/.cd,
                number of strands=2,
                ultra thick,
                gap=0.1,
                name prefix=braid,
            ] {braid={a_{1}^{-1}}};
            \node[] at (-.25,-.12) {$x$};
            \node[] at (1.25,-.12) {$y$};
            \node[] at (-.25,-1.4) {$u$};
            \node[] at (1.25,-1.4) {$v$};
            \node[] at (-.25,-.75) {$r$};
        \end{tikzpicture}
        \hspace{1cm}
        \begin{tikzpicture}
            \pic[braid/.cd,
                number of strands=2,
                ultra thick,
                gap=0.1,
                name prefix=braid,
            ] {braid={a_{1}}};
            \node[] at (-.25,-.12) {$x$};
            \node[] at (1.25,-.12) {$y$};
            \node[] at (-.25,-1.4) {$u$};
            \node[] at (1.25,-1.4) {$v$};
            \node[] at (-.25,-.75) {$r^{-1}$};
        \end{tikzpicture}
        \caption{Any row or column determines the whole square.}
        \label{fig:braid}
    \end{figure}



\subsection{The derived solution}
   
    \begin{proposition}\label{prop:derived}
        Let $(X,r)$ be a non-degenerate set-theoretic solution to the Yang--Baxter equation. For any $x,y\in X$ consider
        \begin{align*}
            \sigma_{y}(x)= \lambda_y\rho_{\lambda_x^{-1}(y)}(x).
        \end{align*}
        Then $s:X\times X\to X \times X$ defined by $s(x,y) =(y,\sigma_y(x))$ is a solution to the Yang-Baxter equation.
        Moreover, $r$ is bijective if and only if $\sigma_y$ is bijective for every $y\in X$.
    \end{proposition}
    
    \begin{exercise}\label{ex:derived}
        Prove Proposition~\ref{prop:derived}.
    \end{exercise}
    
    \begin{convention}
        From now on, a \emph{solution}  will always mean a non-degenerate bijective set-theoretic solution to the Yang--Baxter equation.
    \end{convention}

    \begin{definition}
         Let $(X,r)$ be a solution. The pair $(X,s)$ is called the \emph{derived solution} of the solution $(X,r)$.
    \end{definition}



    \subsection{Involutive solutions}
    In \cite{MR2278047}, Rump proved that radical rings provide examples of involutive solutions.

    \begin{definition}
        A ring $(R,+,\cdot)$ is said to be \emph{radical} if $R$ with respect the binary operation $\circ$ defined by $x\circ y=x+y+xy$ is a group. We denote by $x'$ the inverse of $x$ with respect to $\circ$.  
    \end{definition}
    
    \begin{definition}
        A solution $(X,r)$ is said to be \emph{involutive} if $r^2=\id_{X\times X}$.
    \end{definition}
    
    \begin{proposition}
        Let $R$ be a radical ring. Then $(R,r)$, where 
        \begin{align}\label{eq:solradical}
            r(x, y) = (-x + x\circ y, (-x + x \circ y)'\circ x\circ y)
        \end{align}
        is an involutive solution to the YBE.
    \end{proposition}

    \begin{exercise}\label{ex:involutive}
        Let $X$ be a non-empty set and $\lambda_x:X \to X$ bijective maps. Define $r:X\times X \to X \times X$ by $r(x,y)=(\lambda_x(y),\lambda^{-1}_{\lambda_x(y)}(x))$. Prove that $(X,r)$ is an involutive map satisfying \eqref{eq:YBE} if and only if 
        \begin{align*}
            \lambda_x\lambda_{\lambda^{-1}_x(y)} =\lambda_y\lambda_{\lambda^{-1}_y(x)},
        \end{align*}
        for all $x,y\in X$.
    \end{exercise}


    \subsection{Skew braces}
    Do we need radical rings to produce solutions of the form \eqref{eq:solradical}?

    \begin{definition}
         A \emph{skew (left) brace} is a triple $(A,+,\circ)$, where 
        $(A,+)$ and $(A,\circ)$ 
    	are (not necessarily abelian) 
    	groups and 
    	\begin{align}\label{compatibility}
    	    a\circ(b+c)=(a\circ b)-a+(a\circ c)
    	\end{align}
    	holds for all $a,b,c\in A$. 
    
    The groups $(A,+)$ and $(A,\circ)$ are respectively the \emph{additive} and \emph{multiplicative} group	of the skew brace $A$.
    \end{definition}

    One says that a skew left brace $A$ is of \emph{abelian type} (it also is simply called a left brace) if $(A,+)$ is an abelian group. 
    In general, the properties of the additive group determine the type of a skew brace\footnote{Such terminology is borrowed from Hopf-Galois extension where the additive group determines the type of the extension.}.
    
    \begin{remark}
        Even though we use the additive notation, the group $(A,+)$ is not necessarily abelian.
    \end{remark}

    \begin{convention}
         The identity of $(A,+)$ will be denoted by $0$ and the inverse of an element $a$ will be denoted by $-a$. 

         As for radical rings, we write $a'$ to denote the inverse of $a$ with respect to the circle operation $\circ$. 
    \end{convention}
        
    Right left braces are defined similarly.

    \begin{definition}
         A \emph{skew right brace} is a triple $(A,+,\circ)$, where 
        $(A,+)$ and $(A,\circ)$ 
	   are groups and 
	   \begin{align*}
	       (a+b)\circ c=a\circ c-c+b\circ c.
	   \end{align*}
	   holds for all $a,b,c\in A$. 
    \end{definition}

   \begin{exercise}
        Prove that there exists a bijective correspondence between skew left braces
        and skew right braces.
   \end{exercise}

    \begin{convention}
        From now on, with the term skew brace, we will always mean a skew left brace.
    \end{convention}

    \begin{examples}
        Let $(A,+)$ be a group. 
        \begin{enumerate}
            \item Then $A$ is a skew brace with
            $a\circ b=a+b$ for all $a,b\in A$. Such a skew brace is called a \emph{trival skew brace}.
            \item Similarly, the operation $a\circ b=b+a$ turns $A$ into a skew brace. Such a skew brace is called an \emph{almost trivial skew brace}. 
        \end{enumerate}
    \end{examples}

    \subsection{First examples}

    \begin{example}
        Let $(A,+)$ be a group. Then $A$ is a skew brace with
            $a\circ b=a+b$ for all $a,b\in A$. Such a skew brace is called a \emph{trivial skew brace}.
    \end{example}
    
    \begin{example}
          Let $(A,+)$ be a group. The operation $a\circ b=b+a$ turns $A$ into a skew brace. Such a skew brace is called an \emph{almost trivial skew brace}. 
    \end{example}

    \begin{definition}
        Let $A$ and $B$ be skew braces. Then $A\times B$ with 
    	\begin{align*}
    		(a,b)+(a_1,b_1)&=(a+a_1,b+b_1),\\
    		(a,b)\circ (a_1,b_1)&=(a\circ a_1,b\circ b_1),
    	\end{align*}
    	is a skew brace. This is the \emph{direct product} of the skew braces $A$ and $B$. 
    \end{definition}
    
    \begin{exercise}\label{ex:sd}
    	Let $(A,+)$ and $(M,+)$ be groups and let $\alpha\colon A\to\Aut(M)$ be a
    	group homomorphism. Prove that $M\times A$ with 
    	\begin{align*}
        	(x,a)+(y,b)&=(x+y,a+b),
        	\\
        	(x,a)\circ (y,b)&=(x+\alpha_a(y),a+b)
    	\end{align*}
    	is a skew brace. Similarly, prove that $M\times A$ with
    	\begin{align*}
        	(x,a)+(y,b)&=(x+\alpha_a(y),a+b),\\
        	(x,a)\circ (y,b)&=(x+y,b+a)
    	\end{align*}
    	is a skew brace. 
    \end{exercise}

    \begin{exercise}\label{ex:ef}
        Let $(A,+)$ be a group with an \emph{exact factorisation} through the subgroups $B$ and $C$ (i.e. $B$ and $C$ are subgroups of $A$ such that $B\cap C=\{ 0\}$ and $A=B+C$). 
        This means that each $x\in A$ can be written in a unique way as $x=x_B+x_C$, for some $x_B\in B$ and $x_C\in C$.
        Set
        \begin{align*}
		    x\circ y&=x_B+y+x_C.
	    \end{align*}
        Prove that
        \begin{enumerate}
            \item $(A,\circ)$ is a group isomorphic to $B\times C$, the direct product of $B$ and $C$.
            \item $(A,+,\circ)$ is a  skew brace.
        \end{enumerate}
    \end{exercise}

    
    \subsection{Basic properties of skew braces}

    \begin{lemma}\label{lem:propskew}
        Let $A$ be a skew brace. The following statements hold.
        \begin{enumerate}
            \item $1=0$, where $0$ denotes the identity of $(A,+)$ and by $1$ the identity of $(A,\circ)$.
            \item $-(a\circ b) = - a + a\circ(- b) - a$, for all $a,b \in A$.
        \end{enumerate}
    \end{lemma}
    \begin{proof}
        By \eqref{compatibility} we have
        \begin{align*}
            0= 1\circ 0 = 1 \circ (0+0) = 1\circ 0 -1 + 1 \circ 0 = -1.
        \end{align*}
        Hence $0=1$. 
        Now, let $a,b\in B$. From what we just proved and by \eqref{compatibility}
        we have 
        \begin{align*}
            a = a \circ 0 = a\circ (b-b) = a\circ b - a + a\circ (-b).
        \end{align*}
        and 2) follows. 
    \end{proof}
    
    \begin{proposition}\label{prop:lambda}
        Let $A$ be a skew brace. For each $a\in A$, the map 
        \begin{align*}
            \lambda_a:A\to A, b\mapsto -a + a \circ b
        \end{align*}
    is an automorphism of $(A,+)$. 
    
    Moreover, the map $\lambda: (A,\circ) \to \Aut(A,+), a \mapsto \lambda_a$, is a group homomorphism.
    \end{proposition}

    \begin{proof}
        First, let us prove that $\lambda_a$ is an endomorphism of $(A,+)$, for all $a\in A$.
        We have that
        \begin{align*}
            \lambda_a(b+c) = -a+a\circ (b+c) \overset{\eqref{compatibility}}{=} -a +a\circ b -a +a\circ c,
        \end{align*}
        for all $b,c\in A$. Now, for any $b\in A$,
        \begin{align*}
            \lambda_0(b) = - 0 + 0 \circ b \overset{1=0}{=} b,
        \end{align*}
        hence $\lambda_0=\id_A$. Moreover, for any $a,b,c\in A$, 
        \begin{align*}
            \lambda_a\lambda_b(c) = -a +a\circ(-b+b\circ c) = -a + a\circ(-b) - a + a\circ b \circ c=-(a\circ b) + a\circ b \circ c = \lambda_{a\circ b}(c).
        \end{align*}
        Hence, $\lambda_a\lambda_b=\lambda_{a\circ b}$, for all $a,b\in A$. It follows that for any $a\in A$, the map $\lambda_a$ is bijective with inverse $\lambda_{a'}$. 
    \end{proof}

    \begin{exercise}\label{ex:rho}
        Let $A$ be a skew brace. Prove that 
        \begin{align*}
            a\circ(a'+b) = \lambda_a(b),
        \end{align*}
        for all $a,b\in A$. As a consequence, we have that $\rho_b(a)=(a'+b)'\circ b$, for all $a,b\in A$.
    \end{exercise}

    \begin{proposition}\label{prop:rho}
        Let $A$ be a brace. For each $a\in A$, the map
        \begin{align*}
            \rho_b\colon A\to A,\quad
            \mapsto (\lambda_a(b))'\circ a\circ b,
        \end{align*}
        is bijective. Moreover, the map 
        $\rho\colon (A,\circ)\to\Sym(A)$, $b\mapsto\rho_b$, satisfies $\rho_c\rho_b=\rho_{b\circ c}$, for all $b,c\in A$. 
    \end{proposition}

    \begin{proof}
    By Exercise~\ref{ex:rho}, we get that $\rho_b(a)=(a'+b)'\circ b$, for all $a,b\in A$.
    Now, for all $a\in A$, we have that 
    \begin{align*}
        \rho_0(a) = (a'+0)'\circ 0 = a,
    \end{align*}
    i.e., $\rho_0=\id_A$. Moreover, for all $a,b,c\in A$, we have
    \begin{align*}
        \rho_c\rho_b(a)&=((\rho_b(a))'+c)'\circ c
        = (((a'+b)'\circ b)' + c)' \circ c\\
        &= ((b'\circ (a'+b) + c)'\circ c
        = (b'\circ a' - b + c)' \circ c\\
        &= (b'\circ(a' +b\circ c))'\circ c
        = (a' +b\circ c)'\circ b \circ c\\
        &= \rho_{b\circ c}(a),
    \end{align*}
    i.e., $\rho_{b\circ c} = \rho_c\rho_b$. It also follows that $\rho_b$ is bijective with inverse $\rho_{b'}$ for every $b\in A$.
    \end{proof}

    \subsection{Skew braces and solutions}
    
    Now we can state the theorem that gives a first connection of skew braces with solutions. 
    The following result has been proved by Guarnieri and Vendramin in \cite{MR3647970} extending an analogous result proved by Rump in \cite{MR2278047} for involutive solutions.  

    \begin{theorem}
        Let $A$ be a skew brace. Then $(A,r_A)$, where 
        \begin{align*}
            r_A(x, y) = (-x + x\circ y, (-x + x \circ y)'\circ x\circ y)
        \end{align*}
        is a bijective solution to the YBE.
        Moreover, $(A, r_A)$ is involutive if and only if $A$ is of abelian type.
    \end{theorem}

    \begin{proof}
        As before, let us set 
        \begin{align*}
            \lambda_x(y)&=-x+x\circ y\\
            \rho_y(x) &= (\lambda_x(y))'\circ x \circ y.
        \end{align*}
        By Proposition~\ref{prop:lambda} and Proposition~\ref{prop:rho}, we have that 
        $\lambda: A \to \Aut(A,+)$ is a left action of $A$ on itself and $\rho: A\to \Sym(A)$ is a right action of $A$ on itself. Moreover, by definition
        \begin{align*}
            \lambda_x(y)\circ\rho_y(x)=x\circ y,
        \end{align*}
        i.e. condition~\eqref{eq:LYZ} in Theorem~\ref{thm:LYZ} is satisfied. 
        Hence, by Theorem~\ref{thm:LYZ}, $(A,r_A)$ is a solution.

        Now let us compute $r_A^2$,
        \begin{align*}
            r_A^2(x,y) = (-\lambda_x(y) + \lambda_x(y)\circ \rho_y(x), (-\lambda_x(y) + \lambda_x(y)\circ \rho_y(x))'\circ \lambda_x(y)\circ \rho_y(x)).
        \end{align*}

        First if we assume $(A,+)$ abelian, we have
        \begin{align*}
            -\lambda_x(y) + \lambda_x(y)\circ \rho_y(x) &= -(-x+x\circ y) + x\circ y
            = -(x\circ y) +x +x\circ y \\
            &\overset{\text{Lemma}~\ref{lem:propskew}}{=}
            - x + x\circ (-y) +x\circ y = x\circ (-y)-x +x\circ y \\
            &\overset{\eqref{compatibility}}{=} x\circ(-y+y) = x
        \end{align*}
        and
        \begin{align*}
            (-\lambda_x(y) + \lambda_x(y)\circ \rho_y(x))'\circ \lambda_x(y)\circ \rho_y(x)
            =x' \circ x\circ y = y.
        \end{align*}
        Hence, $(A,r_A)$ is involutive.

        Now let us assume $(A,r_A)$ involutive. In particular, for all $x,y\in A$
        \begin{align*}
            x= -\lambda_x(y) + \lambda_x(y)\circ \rho_y(x) = -(x\circ y) +x +x\circ y.
        \end{align*}
        For the arbitrary of $y$ and since $(A,\circ)$ is a group, it follows
        $x=-y+x+y$, for all $x,y \in A$, i.e. $(A,+)$ is abelian.        
    \end{proof}

    \begin{exercise}\label{ex:sumprod}
        Let $A$ be a skew brace. Prove that 
        \begin{align*}
            a+b = a\circ\lambda^{-1}_{a}(b)
        \end{align*}
        and
        \begin{align*}
            a\circ b = a + \lambda_a(b)
        \end{align*}
    \end{exercise}


\subsection{Subbraces and ideals.}

    \begin{definition}
        Let $A$ be a skew brace. 

        A \emph{subbrace} f $A$ is a subset $B$ of $A$ such
        that $(B,+)$ is a subgroup of $(A,+)$ and $(B,\circ)$ is a subgroup of $(A,\circ)$.

        A \emph{left ideal} of $A$ is a subgroup $(I,+)$ of $(A,+)$ such that $\lambda_b(I) \subseteq I$ for all $b\in B$, i.e. $\lambda_b(x)\in I$ for all $b\in A$ and $x\in I$.

        A \emph{strong left ideal} of $A$ is a left ideal $I$ of $A$ such that $(I,+)$ is a normal subgroup of $(A,+)$.
    \end{definition}

    \begin{lemma}
        A left ideal $I$ of a skew brace $A$ is a subbrace of $B$.
    \end{lemma}

    \begin{proof}
        We need to prove that $(I,\circ)$ is a subgroup of $(A,\circ)$. 
        Clearly $I$ is non-empty, as it is an additive subgroup of $A$. 
        If $x,y\in I$, then 
        \begin{align*}
            x\circ y = x - x+ x\circ y = x+\lambda_x(y) \in I + I = I
        \end{align*}
        and
        \begin{align*}
            x'=-\lambda_{x'}(x) \in I. 
        \end{align*}
    \end{proof}

    \begin{exercise}
    Let $A$ be a skew brace. Then
    \begin{align*}
        \Fix(A) = \{b \in B\colon \lambda_x(b)=b,\ \forall b\in A\}
    \end{align*}
    is a left ideal of $A$.
    \end{exercise}

    \begin{definition}
        An \emph{ideal} of a skew brace $A$ is a strong left ideal $I$ of $A$ such that    $(I,\circ)$ is a normal subgroup of $(A,\circ)$.
    \end{definition}

    In general, left ideals, strong left ideals and ideals are different notions.
    
    \begin{definition}
        Let $A$ be a skew brace. The subset $\Soc(A) = \ker \lambda \cap Z(A, +)$ is the
        \emph{socle} of $A$.
    \end{definition}

    \begin{proof}
        First, $\Soc(A) \neq \empty$, since $0\in \Soc(A)$. Moreover, if $x\in \Soc(A)$, then $x'=\lambda_x(x') = - x$.
        It follows that if $x,y\in \Soc(A)$ then 
        \begin{align*}
            \lambda_{x-y}=\lambda_{x\circ y'} = \lambda_x\lambda_{y'} =\lambda_x\lambda^{-1}_y =\id_A,
        \end{align*}
        and, clearly $x-y\in Z(A,+)$. Hence, $\Soc(A)$ is an additive subgroup of $A$ and since $\Soc(A)$ is a subgroup of $Z(A,+)$ it is also a normal additive subgroup of $A$. 
        Moreover,
        for all $x\in \Soc(A)$ and $a \in A$: 
        \begin{align}
            \label{eq1}\lambda_a(x)=a\circ x - a\\
            \label{eq2}\lambda_a(x)=a\circ x\circ a'.
        \end{align}
        For the first equality we have that applying Exercise~\ref{ex:sumprod}
        \begin{align*}
            \lambda_a(x)= a\circ (a'+x) = a\circ(x+a') \overset{\eqref{compatibility}}{=} a\circ x - a,
        \end{align*}
        for the second equality
        \begin{align*}
            \lambda_a(x)=a\circ (a'+x) =a\circ(x\circ\lambda_x(a')) = a\circ x \circ a'.
        \end{align*}
        It follows that, for all $x\in \Soc(A)$ and $a,b\in A$, we have
        \begin{align*}
            \lambda_{\lambda_a(x)} \overset{\eqref{eq1}}{=} \lambda_{a\circ x \circ a'} = \lambda_a\lambda_x\lambda_a^{-1} = \lambda_a\lambda_a^{-1}=\id_A
        \end{align*}
        and, by Exercise~\ref{ex:sumprod},
        \begin{align*}
            b+\lambda_a(x) &= b\circ\lambda^{-1}_{b}\lambda_a(x) = b \circ\lambda_{b'\circ a}(x) \overset{\eqref{eq2}}{=} a \circ x \circ a \circ b \\&\overset{\eqref{eq2}}{=} \lambda_a(x)\circ b = \lambda_a(x) + \lambda_{\lambda_a(x)}(b) = \lambda_a(x) +b,
        \end{align*}
        i.e., $\lambda_a(x)\in Z(A,+)$. Finally, it also follows that for any $a\in A$ and $x\in \Soc(A)$, $a\circ x\circ a'\in \Soc(A)$. Therefore, $\Soc(A)$ is an ideal of $A$.
    \end{proof}

    \begin{exercise}
        Let $A$ be a skew brace. Prove that $\Soc(A)=\ker \lambda \cap \ker \rho$.
    \end{exercise}

    \begin{definition}
         Let $A$ be a skew brace. The subset $\Ann(A) = \Soc(A) \cap Z(B, \circ)$ is the
        \emph{annihilator} of $B$.
    \end{definition}

    \begin{proposition}
         The annihilator of a skew brace $A$ is an ideal of $A$.
    \end{proposition}

    \begin{proof}
        First, if $x,y\in \Ann(A)$, then $x-y\in \Soc(A)$ and for any $a\in A$
        \begin{align*}
            (x-y)\circ a = x\circ y' \circ a = x\circ a\circ y' a\circ x \circ y' = a\circ (x-y),
        \end{align*}
        i.e. $x-y\in \Ann(A)$. Now, since $\Ann(A)\subseteq Z(A,+)\cap Z(A,\circ)$, we only need to prove $\lambda_a(x)\in \Ann(A)$, for all $x\in \Ann(A)$ and $a\in A$. By \eqref{eq1} we have that $\lambda_a(x)=a\circ x\circ a'= x\circ a \circ a' = x\in \Ann(A)$. 
    \end{proof}

    \subsection{The isomorphism theorems}

    If $A$ is a skew brace and $I$ is an ideal of $A$, then $a+I = a\circ I$ for all $a\in A$. 
    
    This allows us to prove that there exists a unique skew brace structure over $A/I$ such that the map
    \begin{align*}
        A\mapsto A/I, \quad a\mapsto a+I=a\circ I,
    \end{align*}
    is a homomorphism of skew braces. 
    
    \begin{definition}
        The skew brace $A/I$ is the \emph{quotient skew brace} of $A$ modulo $I$. 
    \end{definition}
    
    
    It is possible to prove the isomorphism theorems for skew braces. (See Exercises~\ref{ex:thmiso1}--\ref{ex:thmiso4}).
