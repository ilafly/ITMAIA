\section{21/02/2024}

\subsection{The Yang--Baxter equation}
The Yang-Baxter equation (YBE) is one important equation in mathematical physics.
It first appeared in two independent papers of Yang \cite{Yang1967} and Baxter\
\cite{Baxter1971}.

\begin{definition}\index{Yang--Baxter equation}
    A solution of the \emph{Yang--Baxter eqution} is a bijective linear map  $R: V\otimes V \to V \otimes V$, where $V$ is a vector space such that
    \begin{align*}
        R_{12}R_{13}R_{23} = R_{23}R_{13}R_{12}
    \end{align*}
    where $R_{ij}$ denotes the map $V\otimes V\otimes V \to V \otimes V \otimes V$ acting as $R$ on the $(i,j)$
    factor and as the identity on the remaining factor.
\end{definition}
    Let $\tau: V\otimes V \to V \otimes V$ be the map $\tau(u\otimes v) = v \otimes u$ for $u,v \in V$.
    It's easy to check (try!) that $R: V \otimes V \to V \otimes V$ is a solution of the Yang--Baxter equation if and only if $\bar{R}:=\tau R$ satisfies 
    \begin{align*}
        \bar{R}_{12}\bar{R}_{23}\bar{R}_{12}=\bar{R}_{23}\bar{R}_{12}\bar{R}_{23}.
    \end{align*}

    An interesting class of solutions of the Yang--Baxter equation arises when $V$ has a $R$-invariant basis $X$. In such a case the solution is said to be set-theoretic. 

\subsection{The set-theoretic version}
    Drinfeld in \cite{Dri1992} observed it makes sense to consider the Yang--Baxter equation in the category of sets and stated that 

    
    \begin{quote}\textit{it would be interesting to study set-theoretic solutions.}\end{quote}
    

    These lectures will focus on set-theoretic solutions to the Yang-Baxter equation and their connection with known and ``new'' algebraic structures. 

    \begin{definition}\index{Set-theoretic solution}
        A \emph{set-theoretic solution to the Yang--Baxter equation} is a pair $(X,r)$ where $X$ is a non-empty set and $r: X\times X \to X \times X$ is a bijective map such that
        \begin{align}\label{eq:YBE}
            (r\times \id)(\id \times r)(r\times \id) = (\id \times r)(r\times \id)(\id \times r)
        \end{align}
    \end{definition}
    
    \begin{convention}
        If $(X,r)$ is a set-theoretic solution to the Yang--Baxter equation, we write 
        \begin{align*}
            r(x,y) = (\lambda_x(y),\rho_y(x))        
        \end{align*}
        where $\lambda_x,\rho_x:X\to X$.
    \end{convention}


    \begin{definition}\index{Set-theoretic solution!Finite}\index{Set-theoretic solution!Non-degenerate}
        Let $(X,r)$ be a set-theoretic solution to the  Yang--Baxter equation. We say that 
        \begin{itemize}
            \item $(X,r)$ is \emph{finite} if $X$ is finite.
            \item $(X,r)$ is \emph{non-degenerate} if $\lambda_x,\rho_x$ are bijective for all $x\in X$.
        \end{itemize}
    \end{definition}

\subsection{Set-theoretic solutions to the Yang--Baxter equation and III Reidemeister move}
    Let us represent the map $r:X \times X \to X \times X$ as a crossing and the identity on $X$ as a straight line, see Figure~\ref{fig:crossing}. \index{III Reidemeister move}
    \begin{figure}[h!]
        \centering
        \begin{tikzpicture}
            \pic[braid/.cd,
                number of strands=3,
                ultra thick,
                gap=0.1,
                name prefix=braid,
            ] {braid={a_1}};
        \end{tikzpicture}
        \caption{The map $r$ represented as a crossing and the identity as a straight line.}
        \label{fig:crossing}
    \end{figure}

    Then the Yang--Baxter equation can be pictured as in Figure~\ref{fig:YangBaxter}.

    \begin{figure}
        \centering
        \begin{tikzpicture}
            \pic[braid/.cd,
                number of strands=3,
                ultra thick,
                gap=0.1,
                name prefix=braid,
            ] {braid={a_{1}a_2a_1}};
            \node[font=\Large] at (3,-2) {\(=\)};
        \end{tikzpicture}
        \hspace{.5cm}
        \begin{tikzpicture}
        \pic[braid/.cd,
            number of strands=3,
            ultra thick,
            gap=0.1,
            name prefix=braid,
        ] {braid={a_{2}a_1a_2}};
        \end{tikzpicture}
        \caption{The Yang--Baxter equation.}
        \label{fig:YangBaxter}
    \end{figure}

    Moreover, under the assumption of $(X,r)$ being non-degenerate we have the following lemma. 
    
    \begin{lemma}
    \label{lem:LYZ}
        Let $(X,r)$ be a solution to the Yang--Baxter equation. 
        \begin{enumerate}
            \item Given $x,u\in X$, there exist unique $y,v\in X$ such that $r(x,y)=(u,v)$. 
            \item Given $y,v\in X$, there exist unique $x,u\in X$ such that $r(x,y)=(u,v)$. 
        \end{enumerate}
    \end{lemma}
    \begin{proof}
        For the first claim take $y=\lambda_x^{-1}(u)$ and $v=\rho_y(x)$. 
        For the second, $x=\rho_y^{-1}(v)$ and $u=\lambda_x(y)$. 
    \end{proof}

    So, the bijectivity of $r$ means that any row in Figure~\ref{fig:braid} determines the whole square. By Lemma~\ref{lem:LYZ} we have that non-degeneracy means that any column in Figure~\ref{fig:braid} also determines the whole square. 

    \begin{figure}[h!]
    \centering
        \begin{tikzpicture}
            \pic[braid/.cd,
                number of strands=2,
                ultra thick,
                gap=0.1,
                name prefix=braid,
            ] {braid={a_{1}^{-1}}};
            \node[] at (-.25,-.12) {$x$};
            \node[] at (1.25,-.12) {$y$};
            \node[] at (-.25,-1.4) {$u$};
            \node[] at (1.25,-1.4) {$v$};
            \node[] at (-.25,-.75) {$r$};
        \end{tikzpicture}
        \hspace{1cm}
        \begin{tikzpicture}
            \pic[braid/.cd,
                number of strands=2,
                ultra thick,
                gap=0.1,
                name prefix=braid,
            ] {braid={a_{1}}};
            \node[] at (-.25,-.12) {$x$};
            \node[] at (1.25,-.12) {$y$};
            \node[] at (-.25,-1.4) {$u$};
            \node[] at (1.25,-1.4) {$v$};
            \node[] at (-.25,-.75) {$r^{-1}$};
        \end{tikzpicture}
        \caption{Any row or column determines the whole square.}
        \label{fig:braid}
    \end{figure}

\subsection{First examples}
    \begin{examples}
        Let $X$ be a non-empty set.  
        \begin{enumerate}
            \item The pair $(X,\id_{X\times X})$ is a  set-theoretic solution to the  Yang--Baxter equation. Note that $(X,\id_{X\times X})$ is not non-degenerate, since $\lambda_x(y)=x$ and $\rho_y(x)=y$, for all $x,y\in X$.
            \item Let $\tau: X \times X \to X \times X$ be the flip map, i.e. $\tau(x,y)=(y,x)$ for all $x,y \in X$. Then, the pair $(X,\tau)$ is a set-theoretic solution to the Yang--Baxter equation. Moreover, it is non-degenerate since $\lambda_x=\rho_x=\id_X$ for all $x\in X$.
            \item Let $\lambda, \rho$ be permutaions of $X$. Then $r(x,y)=(\lambda(y),\rho(x))$ is a
            non-degenerate set-theoretic solution to the Yang--Baxter equation if and only if $\lambda\rho = \rho\lambda$.
            Morever, $(X , r )$ is involutive if and only if $\rho = \lambda^{-1}$. The solution 
            $(X,r)$ is called a \emph{permutational solution} or a \emph{Lyubashenko's solution}.
        \end{enumerate}
    \end{examples}

    If on the set $X$ we have a bit more structure we can define some more sophisticated solutions.
    \begin{example}
        Let $G$ be a group and let
        \begin{align*}
            r_1(x,y) &= (y, y^{-1}xy)\\
            r_2(x,y) &= (x^{2}y, y^{-1}x^{-1}y).
        \end{align*}
        Then $(X,r_1)$ and $(X,r_2)$ are bijective non-degenerate set-theoretic solutions to the Yang--Baxter equation.
    \end{example}

    \subsection{A characterisation}
    \begin{proposition}\label{prop:characterisation}
        Let $X$ be a non-empty set and $r: X \times X \to X \times X$ be a  map, written ase $r(x,y) = (\lambda_x(y),\rho_y(x))$. Then $r$ satisfies equation~\ref{eq:YBE} if and only if
    \begin{enumerate}
        \item $\lambda_x\lambda_y = \lambda_{\lambda_x(y)}\lambda_{\rho_y(x)}$
        \item $\lambda_{\rho_{\lambda_y(z)}(x)}\rho_z(y)=\rho_{\lambda_{\rho_y(x)}(z)}\lambda_x(y)$
        \item $\rho_z\rho_y=\rho_{\rho_z(y)}\rho_{\lambda_y(z)}$
    \end{enumerate}
    for all $x,y,z\in X$.

    In particular, $(X,r)$ is a solution to the Yang--Baxter equation when $r$ is bijective.
    \end{proposition}
    \begin{proof}
        Let us write $r_1=r\times \id$ and $r_2=\id \times r$. Then  
        \begin{align*}
            r_1r_2r_1(x,y,z)& = r_1r_2(\lambda_x(y),\rho_y(x),z) \\
            &= r_1(\lambda_x(y),\lambda_{\rho_y(x)}(z),\rho_z\rho_y(x))\\
            &= (\lambda_{\lambda_x(y)}\lambda_{\rho_y(x)}(z),\rho_{\lambda_{\rho_y(x)}(z)}\lambda_x(y),\rho_z\rho_y(x)),
        \end{align*}
        and 
        \begin{align*}
            r_2r_1r_2(x,y,z) =& r_2r_1(x,\lambda_y(z),\rho_z(y)) \\
            &= r_2(\lambda_x\lambda_y(z),\rho_{\lambda_y(z)}(x),\rho_z(y))\\
            &= (\lambda_x\lambda_y(z),\lambda_{\rho_{\lambda_y(z)}(x)}\rho_z(y),\rho_{\rho_z(y)}\rho_{\lambda_y(z)}(x)).
        \end{align*}
        Therefore $r_1r_2r_1= r_2r_1r_2$ if and only if $1), 2)$ and $3)$ hold.     
    \end{proof}

    \begin{exercise}
        Let $(X,r)$ be a set-theoretic solution to the Yang--Baxter equation. Define for all $x,y \in X$
        \begin{align*}
            \bar{r}(x,y) = \tau r \tau (x,y) = (\rho_x(y),\lambda_y(x)).
        \end{align*}
        Then $(X,\bar{r})$ is a set-theoretic solution to the Yang--Baxter equation.
    \end{exercise}

    \subsection{Shelfs and racks}\mbox{}
    
    \begin{exercise}\label{ex2}
         Let $X$ be a non-empty set.
         Let $\triangleleft: X\times X \to X$ be a binary operation and define $r: X\times X \to X\times X$ such that $r(x,y)= (y,x \triangleleft y)$. Then $r$ satisfies equation~\ref{eq:YBE} if and only if $(x\triangleleft y)\triangleleft z=(x\triangleleft z)\triangleleft(y\triangleleft z)$ holds for all $x,y,z \in X$. 
         Moreover, $r$ is bijective if and only if the maps $\rho_y:X\to X, x \mapsto x\triangleleft y$ are bijective.
    \end{exercise}

    \begin{definition}\index{Shelf}\index{Rack}
         A \emph{(right) shelf} is a pair $(X,\triangleleft)$ where $X$ is a non-empty set and $\triangleleft$ is a binary operation such that 
        \begin{align*}
            (x\triangleleft y)\triangleleft z=(x\triangleleft z)\triangleleft(y\triangleleft z).
        \end{align*}
        If, in addition, the maps $\rho_y:X \to X, x \mapsto x\triangleleft y$ are bijective for all $y\in X$, then $(X,\triangleleft)$ is called a \emph{(right) rack}.
    \end{definition}

    \begin{proposition}
        Let $X$ be a non-empty set with a binary operation $\triangleleft: X\times X \to X$. Then $r(x,y) =(y,x\triangleleft y)$ is a set-theoretic solution to the Yang--Baxter equation if and only if $(X,\triangleleft)$ is a rack. 
    \end{proposition}

    \begin{proof}
        Follows from exercise~\ref{ex2}.
    \end{proof}

    \begin{exercise}
         Let $G$ be a group. Then $(G,r)$ where $r(x,y)=(y, y^{-1}xy)$ is a 
         non-degenerate set-theoretic solution to the Yang--Baxter equation.
    \end{exercise}
    
    \begin{convention}
        From now on, a \emph{solution}  will always mean a non-degenerate set-theoretic solution to the Yang--Baxter equation.
    \end{convention}

\subsection{An intriguing connection between group actions and solutions}
    The following theorem is the core result of the paper \cite{LYZ00} by Lu, Yan Zhu.

    \begin{theorem}\label{thm:LYZ}
         Let $G$ be a group, let $\lambda: G\times G \to G, (x,y)\mapsto \lambda_x(y)$ a left group action of $G$ on itself as a set and $\rho: G\times G \to G, (x,y)\mapsto \rho_y(x)$ a right group action of $G$ on itself as a set. If the ``compatibility'' condition 
        \begin{align}\label{eq:LYZ}
            uv = \lambda_u(v)\rho_v(u)
        \end{align}
        holds, then $(G,r)$, where
        \begin{align*}
            r:G\times G \to G\times G, \qquad (x,y)\mapsto (\lambda_x(y),\rho_y(x))
        \end{align*}
        is a solution.
    \end{theorem}

    \begin{exercise}\label{ex:LYZ}
        Prove Theorem~\ref{thm:LYZ}
    \end{exercise}

% \section{Bijective non-degenerate set-theoretic solutions to the YBE.}

% \begin{frame}{Convention}

%      From now on, a \alert{solution} (to the YBE) will always mean a bijective non-degenerate set-theoretic solution to the YBE.
     
% \end{frame}


% \subsection{An intriguing connection with group actions.}

% \begin{frame}
% \frametitle{Main result}
% \framesubtitle{An intriguing connection between group actions and solutions}

%    
% \end{frame}


% \begin{frame}{Proof.}
%     We are left to prove that $r$ is bijective. 
%     We will see this in the next lecture.
% \end{frame}

% % \begin{frame}{Proof.}
% %    Prove $r$ is bijective. 

% %     \vspace{6cm}
% % \end{frame}

% % \begin{frame}{Proof.}
% %    Define $r'(x,y)=((\rho_{x^{-1}}(y^{-1}))^{-1},(\lambda_{y^{-1}}(x^{-1}))^{-1})$. 

% %    Prove $rr'=\id_{X\times X} = r'r$.

% %     \vspace{6cm}
% % \end{frame}

% \begin{frame}{Some Exercises}
%     \begin{enumerate}
%         \item Let $G$ be a group. Prove that the following maps satisfy the set-theoretic YBE.
%         \begin{enumerate}
%             \item $r(x,y) = (y, x^{-1})$.
%             \item $r(x,y) = (y^{-1}, x^{-1})$.
%             \item $r(x,y) = (x^2y,y^{-1}x^{-1}y)$.
%         \end{enumerate}
%         \item Let $(X,r)$ be a set-theoretic solution to the YBE. Prove that $X$ with the map $\tau r \tau$ , where $\tau:X\times X, (x,y)\mapsto (y,x)$, is a set-theoretic solution to the YBE.
%         \item Let $G$ be a group. Prove that the following maps satisfy the set-theoretic YBE.
%         \begin{enumerate}
%             \item $r(x,y) = (x^myx^{-m}, x)$, for any integer $m$.
%             \item $r(x,y) = (xy^{-1}x,x)$.
%         \end{enumerate}
%     \end{enumerate}
% \end{frame}

% \end{document}

