\section{21/02/2024}

\subsection{The Yang--Baxter equation}
The Yang-Baxter equation is one important equation in mathematical physics.
It first appeared in two independent papers of Yang \cite{Yang1967} and Baxter\
\cite{Baxter1971}.

\begin{definition}
    A solution of the \emph{Yang--Baxter eqution} is a linear map  $R: V\otimes V \to V \otimes V$, where $V$ is a vector space such that
    \begin{align*}
        R_{12}R_{13}R_{23} = R_{23}R_{13}R_{12}
    \end{align*}
    where $R_{ij}$ denotes the map $V\otimes V\otimes V \to V \otimes V \otimes V$ acting as $R$ on the $(i,j)$
    factor and as the identity on the remaining factor.
\end{definition}
    Let $\tau: V\otimes V \to V \otimes V$ be the map $\tau(u\otimes v) = v \otimes u$ for $u,v \in V$.
    It's easy to check (try!) that $R: V \otimes V \to V \otimes V$ is a solution of the Yang--Baxter equation if and only if $\bar{R}:=\tau R$ satisfies 
    \begin{align*}
        \bar{R}_{12}\bar{R}_{23}\bar{R}_{12}=\bar{R}_{23}\bar{R}_{12}\bar{R}_{23}.
    \end{align*}

    An interesting class of solutions of the Yang--Baxter equation arises when $V$ has a $R$-invariant basis $X$. In such a case the solution is said to be set-theoretic. 

\subsection{The set-theoretic version}
    Drinfeld in \cite{Dri1992} observed it makes sense to consider the Yang--Baxter equation in the category of sets and stated that 

    
    \begin{quote}\textit{it would be interesting to study set-theoretic solutions.}\end{quote}
    

    These lectures will focus on set-theoretic solutions to the Yang-Baxter equation and their connection with known and ``new'' algebraic structures. 

    \begin{definition}
        A \emph{set-theoretic solution (to the Yang--Baxter equation)} is a pair $(X,r)$ where $X$ is a non-empty set and $r: X\times X \to X \times X$ is a map such that
        
        \begin{align*}
            (r\times \id)(\id \times r)(r\times \id) = (\id \times r)(r\times \id)(\id \times r)
        \end{align*}
    \end{definition}
    
    \begin{convention}
        If $(X,r)$ is a set-theoretic solution to the YBE, we write 
        \begin{align*}
            r(x,y) = (\lambda_x(y),\rho_y(x))        
        \end{align*}
        where $\lambda_x,\rho_x:X\to X$.
    \end{convention}


    \b
        \begin{itemize}
            \item $(X,r)$ is \emph{bijective} if $r$ is bijective.
            \item $(X,r)$ is \emph{finite} if $X$ is finite.
            \item $(X,r)$ is \emph{non-degenerate} if $\lambda_x,\rho_x$ are bijective for all $x\in X$.
        \end{itemize}



% \subsection{A graphical representation.}
% \begin{frame}{Set-theoretic solutions and III Reidemeister move}
% \end{frame}

% \subsection{First examples.}
% \begin{frame}{First examples}
%     Let $X$ be a non-empty set.
%     \bigskip
    
%     \begin{itemize}
%         \item Then $(X,\id_{X\times X})$ is a bijective degenerate (i.e. not non-degenerate) set-theoretic solution to the YBE.
%         \item[]
%         \item Then $(X,\tau)$, where $\tau(x,y)=(y,x)$, is a bijective non-degenerate set-theoretic solution to the YBE.
%     \end{itemize}
% \end{frame}

% \subsection{A characterisation.}
% \begin{frame}{A characterisation}
%     Let $X$ be a non-empty set and $r: X \times X \to X \times X$ be a map and write $r(x,y) = (\lambda_x(y),\rho_y(x))$. Then $(X,r)$ is a set-theoretic solution to the YBE if and only if
%     \begin{itemize}
%         \item $\lambda_x\lambda_y = \lambda_{\lambda_x(y)}\lambda_{\rho_y(x)}$
%         \item $\lambda_{\rho_{\lambda_y(z)}(x)}\rho_z(y)=\rho_{\lambda_{\rho_y(x)}(z)}\lambda_x(y)$
%         \item $\rho_z\rho_y=\rho_{\rho_z(y)}\rho_{\lambda_y(z)}$
%     \end{itemize}
%     for all $x,y,z\in X$.
% \end{frame}

% \begin{frame}
% \frametitle{A characterisation}
% \framesubtitle{Proof}
%     \begin{itemize}
%         \item Write $r_1=r\times \id$ and $r_2=\id \times r$.
%         \item Compute $r_1r_2r_1(x,y,z)$. 
%         \begin{align*}
%             r_1r_2r_1(x,y,z)& = r_1r_2(\lambda_x(y),\rho_y(x),z) \\
%             &= r_1(\lambda_x(y),\lambda_{\rho_y(x)}(z),\rho_z\rho_y(x))\\
%             &= (\lambda_{\lambda_x(y)}\lambda_{\rho_y(x)}(z),\rho_{\lambda_{\rho_y(x)}(z)}\lambda_x(y),\rho_z\rho_y(x))
%         \end{align*}
%         \item Compute $r_2r_1r_2(x,y,z)$.
%         \begin{align*}
%             r_2r_1r_2(x,y,z) =& r_2r_1(x,\lambda_y(z),\rho_z(y)) \\
%             &= r_2(\lambda_x\lambda_y(z),\rho_{\lambda_y(z)}(x),\rho_z(y))\\
%             &= (\lambda_x\lambda_y(z),\lambda_{\rho_{\lambda_y(z)}(x)}\rho_z(y),\rho_{\rho_z(y)}\rho_{\lambda_y(z)}(x))
%         \end{align*}
%         \item Compare.
%     \end{itemize}
% \end{frame}

% \begin{frame}{Example/Exercise 1}


%     Let $(X,r)$ be a set-theoretic solution to the YBE. Define for all $x,y \in X$

%     \begin{align*}
%         \bar{r}(x,y) = \tau r \tau (x,y) = (\rho_x(y),\lambda_y(x)).
%     \end{align*}
%     Then $(X,\bar{r})$ is a set-theoretic solution to the YBE.
    
% \end{frame}

% \begin{frame}{Example/Exercise 2}  
    
%     Let $X$ be a non-empty set.
        
%     Let $\triangleleft: X\times X \to X$ be a binary operation and define $r:X\times X \to X\times X$ such that $r(x,y)= (y,x \triangleleft y)$. Then $(X,r)$ is a set-theoretic solution to the YBE if and only if $(x\triangleleft y)\triangleleft z=(x\triangleleft z)\triangleleft(y\triangleleft z)$. 

%     \bigskip\pause

    
%     \textbf{Proof.} Write $\lambda_x = \id_X$ and $\rho_y(x)=x \triangleleft y$.

%     By the characterisation,
%     \begin{itemize}
%         \item $\lambda_x\lambda_y = \lambda_{\lambda_x(y)}\lambda_{\rho_y(x)}$ \hfill {\large \alert{\checkmark}}\qquad
%         \item $\lambda_{\rho_{\lambda_y(z)}(x)}\rho_z(y)=\rho_{\lambda_{\rho_y(x)}(z)}\lambda_x(y)$
%         \item[] \textcolor{gray}{i.e., $\rho_z(y)=\rho_{z}(y)$}\hfill {\large \alert{\checkmark}}\qquad
%         \item $\rho_z\rho_y=\rho_{\rho_z(y)}\rho_{\lambda_y(z)}$
%         \item[] \textcolor{gray}{i.e., $\rho_z\rho_y(x)=\rho_{\rho_z(y)}\rho_{\lambda_y(z)}(x)$}
%         \item[] \textcolor{gray}{i.e., $(x\triangleleft y)\triangleleft z=(x\triangleleft z)\triangleleft(y\triangleleft z)$}\hfill {\large \alert{\checkmark}}\qquad
%     \end{itemize}   
% \end{frame}

% \begin{frame}{Racks}
%     A \alert{(right) shelf} is a pair $(X,\triangleleft)$ where $X$ is a non-empty set and $\triangleleft$ is a binary operation such that 
%     \begin{align*}
%         (x\triangleleft y)\triangleleft z=(x\triangleleft z)\triangleleft(y\triangleleft z).
%     \end{align*}
%     If, in addition, the maps $\rho_y:X \to X, x \to x\triangleleft y$ are bijective for all $y\in X$, then $(X,\triangleleft)$ is called a \alert{(right) rack}.

% \end{frame}

% \begin{frame}
%     With the previous example, we proved the following.
%     \medskip

%     \textbf{Proposition. } Let $X$ be a non-empty set with a binary operation $\triangleleft: X\times X \to X$. Then $r(x,y) =(y,x\triangleleft y)$ is a set-theoretic solution to the YBE if and only if $(X,\triangleleft)$ is a shelf. 
    
%     Moreover, $(X,r)$ is bijective and non-degenerate if and only if $(X,r)$ is a rack. \alert{(prove it.)}

% \end{frame}

% \begin{frame}{Example/Exercise 3}
%     Let $X$ be a non-empty set.
%     \bigskip

%     \begin{itemize}
%         \item Let $\lambda,\rho:X \to X$ and define $r:X\times X \to X\times X$ such that $r(x,y)= (\lambda(y),\rho(x))$. Then $(X,r)$ is a set-theoretic solution to the YBE if and only if $\lambda\rho = \rho\lambda$.\footnote{When $\lambda$ and $\rho$ are bijective the solution is called the \alert{permutation solution} associated with the permutations $\lambda$ and $\rho$.} 
%     \end{itemize}
%     \bigskip

%     \pause
%     \remarkbox[ExeterLightYellow]{\textcolor{ExeterGreen}{
%     Why? 
%     Can you say when $(X,r)$ is bijective? 
%     What about non-degenerate?}}
% \end{frame}

% \begin{frame}{Example/Exercise 4}

%     Let $G$ be a group. Then $(G,r)$ where $r(x,y)=(y, y^{-1}xy)$ is a bijective non-degenerate set-theoretic solution to the YBE. \\(\alert{Prove it}).

%     \bigskip\pause
    
%     \textbf{Hint. } Prove that $G$ with the binary operation defined by $x\triangleleft y = y^{-1}xy$ is a rack.
% \end{frame}

% \section{Bijective non-degenerate set-theoretic solutions to the YBE.}

% \begin{frame}{Convention}

%      From now on, a \alert{solution} (to the YBE) will always mean a bijective non-degenerate set-theoretic solution to the YBE.
     
% \end{frame}


% \subsection{An intriguing connection with group actions.}

% \begin{frame}
% \frametitle{Main result}
% \framesubtitle{An intriguing connection between group actions and solutions}

%     Let $G$ be a group, let $\lambda: G\times G \to G, (x,y)\mapsto \lambda_x(y)$ a left group action of $G$ on itself as a set and $\rho: G\times G \to G, (x,y)\mapsto \rho_y(x)$ a right group action of $G$ on itself as a set. If the ``compatibility'' condition 
%     \begin{align*}
%         uv = \lambda_u(v)\rho_v(u)
%     \end{align*}
%     holds, then $(G,r)$, where
%     \begin{align*}
%         r:G\times G \to G\times G, \qquad (x,y)\mapsto (\lambda_x(y),\rho_y(x))
%     \end{align*}
%     is a solution to the YBE.\footfullcite{LYZ00}
% \end{frame}

% \begin{frame}{Proof.}
%     \begin{itemize}
%         \item Write $r_1=r\times \id$ and $r_2=\id \times r$.
%         \item Write 
%         \begin{align*}
%             r_1r_2r_1(x,y,z)&= (\lambda_{\lambda_x(y)}\lambda_{\rho_y(x)}(z),\rho_{\lambda_{\rho_y(x)}(z)}\lambda_x(y),\rho_z\rho_y(x)) \\&=(u_1,v_1,w_1).
%         \end{align*}
%         \item Write 
%         \begin{align*}
%             r_2r_1r_2(x,y,z)&=(\lambda_x\lambda_y(z),\lambda_{\rho_{\lambda_y(z)}(x)}\rho_z(y),\rho_{\rho_z(y)}\rho_{\lambda_y(z)}(x))\\&=(u_2,v_2,w_2).
%         \end{align*}
%     \end{itemize}
% \end{frame}

% \begin{frame}{Proof.}
%     $(u_1,v_1,w_1)= (\lambda_{\lambda_x(y)}\lambda_{\rho_y(x)}(z),\rho_{\lambda_{\rho_y(x)}(z)}\lambda_x(y),\rho_z\rho_y(x))$

%     $(u_2,v_2,w_2)=(\lambda_x\lambda_y(z),\lambda_{\rho_{\lambda_y(z)}(x)}\rho_z(y),\rho_{\rho_z(y)}\rho_{\lambda_y(z)}(x))$

%     \vspace{6cm}

% \end{frame}

% \begin{frame}{Proof.}
%     $(u_1,v_1,w_1)= (\lambda_{\lambda_x(y)}\lambda_{\rho_y(x)}(z),\rho_{\lambda_{\rho_y(x)}(z)}\lambda_x(y),\rho_z\rho_y(x))$

%     $(u_2,v_2,w_2)=(\lambda_x\lambda_y(z),\lambda_{\rho_{\lambda_y(z)}(x)}\rho_z(y),\rho_{\rho_z(y)}\rho_{\lambda_y(z)}(x))$

%     \vspace{6cm}

% \end{frame}

% \begin{frame}{Proof.}
%     We know $u_1v_1w_1=xyz=u_2v_2w_2$.
    
%     $(u_1,v_1,w_1)= (\lambda_{\lambda_x(y)}\lambda_{\rho_y(x)}(z),\rho_{\lambda_{\rho_y(x)}(z)}\lambda_x(y),\rho_z\rho_y(x))$

%     $(u_2,v_2,w_2)=(\lambda_x\lambda_y(z),\lambda_{\rho_{\lambda_y(z)}(x)}\rho_z(y),\rho_{\rho_z(y)}\rho_{\lambda_y(z)}(x))$

%     \vspace{6cm}

% \end{frame}

% \begin{frame}{Proof.}
%     We know $u_1v_1w_1=xyz=u_2v_2w_2$.
    
%     $(u_1,v_1,w_1)= (\lambda_{\lambda_x(y)}\lambda_{\rho_y(x)}(z),\rho_{\lambda_{\rho_y(x)}(z)}\lambda_x(y),\rho_z\rho_y(x))$

%     $(u_2,v_2,w_2)=(\lambda_x\lambda_y(z),\lambda_{\rho_{\lambda_y(z)}(x)}\rho_z(y),\rho_{\rho_z(y)}\rho_{\lambda_y(z)}(x))$

%     \vspace{6cm}

% \end{frame}

% \begin{frame}{Proof.}
%     We know 
    
%     $u_1v_1w_1=xyz=u_2v_2w_2$,
    
%     $u_1=u_2$,

%     $w_1 =w_2$.

%      $(u_1,v_1,w_1)= (\lambda_{\lambda_x(y)}\lambda_{\rho_y(x)}(z),\rho_{\lambda_{\rho_y(x)}(z)}\lambda_x(y),\rho_z\rho_y(x))$

%     $(u_2,v_2,w_2)=(\lambda_x\lambda_y(z),\lambda_{\rho_{\lambda_y(z)}(x)}\rho_z(y),\rho_{\rho_z(y)}\rho_{\lambda_y(z)}(x))$
%     \vspace{6cm}

% \end{frame}

% \begin{frame}{Proof.}
%     We are left to prove that $r$ is bijective. 
%     We will see this in the next lecture.
% \end{frame}

% % \begin{frame}{Proof.}
% %    Prove $r$ is bijective. 

% %     \vspace{6cm}
% % \end{frame}

% % \begin{frame}{Proof.}
% %    Define $r'(x,y)=((\rho_{x^{-1}}(y^{-1}))^{-1},(\lambda_{y^{-1}}(x^{-1}))^{-1})$. 

% %    Prove $rr'=\id_{X\times X} = r'r$.

% %     \vspace{6cm}
% % \end{frame}

% \begin{frame}{Some Exercises}
%     \begin{enumerate}
%         \item Let $G$ be a group. Prove that the following maps satisfy the set-theoretic YBE.
%         \begin{enumerate}
%             \item $r(x,y) = (y, x^{-1})$.
%             \item $r(x,y) = (y^{-1}, x^{-1})$.
%             \item $r(x,y) = (x^2y,y^{-1}x^{-1}y)$.
%         \end{enumerate}
%         \item Let $(X,r)$ be a set-theoretic solution to the YBE. Prove that $X$ with the map $\tau r \tau$ , where $\tau:X\times X, (x,y)\mapsto (y,x)$, is a set-theoretic solution to the YBE.
%         \item Let $G$ be a group. Prove that the following maps satisfy the set-theoretic YBE.
%         \begin{enumerate}
%             \item $r(x,y) = (x^myx^{-m}, x)$, for any integer $m$.
%             \item $r(x,y) = (xy^{-1}x,x)$.
%         \end{enumerate}
%     \end{enumerate}
% \end{frame}

% \end{document}

